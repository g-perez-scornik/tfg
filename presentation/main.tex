\documentclass{beamer}

\usecolortheme[RGB={39,118,193}]{structure}
\usepackage{amsthm}
\newtheorem{axiom}{Axiom}[chapter]
\setcounter{axiom}{-1}
\newtheorem{klaxiom}{KL Axiom}

\newtheorem{proposition}{Proposition}[chapter]
\newtheorem{lemma}[proposition]{Lemma}

\theoremstyle{definition}
\newtheorem{defn}[proposition]{Definition}
\newtheorem{exmp}[proposition]{Example}

\numberwithin{section}{chapter}
\numberwithin{equation}{chapter}

\mathchardef\mhyph="2D
\newcommand{\abs}[1]{\lvert #1 \rvert} %Absolute value
\newcommand{\ddx}[2]{\frac{\partial #1}{\partial #2}} %partial derivative
\newcommand{\ddxk}[4]{\frac{\partial^{#1} #2}{\partial #3 \cdots \partial #4}} %k-th partial derivative
\newcommand{\ddxII}[3]{\frac{\partial^{2} #1}{\partial #2 \partial #3}} %2nd partial derivative
\newcommand{\farg}{-} %empty argument in function
\newcommand{\quot}[2]{#1/#2} %quotient of groups, modules, etc.
\newcommand{\sdgE}{\mathcal{E}}
\newcommand{\Q}{\mathbb{Q}}
\newcommand{\R}{\mathbb{R}}
\newcommand{\vfld}[1]{\mathfrak X(#1)}
\newcommand{\lie}{\mathcal L}
%TODO: make this look better
\newcommand{\qalg}{\Q\mhyph\mathbf{Alg}}
\newcommand{\walg}{\mathbf W}
\newcommand{\mlin}{\mathbf{Mic}}
\newcommand{\vbun}{\mathbf{Fib}}
\renewcommand{\vec}{\underline} %markup for vectors
\DeclareMathOperator{\spn}{span}
\DeclareMathOperator{\spec}{Spec}
\DeclareMathOperator{\rspec}{Spec_R}
\DeclareMathOperator{\id}{id}
\DeclareMathOperator{\aut}{Aut}
\DeclareMathOperator{\der}{Der}



\mode<presentation>
{
  \usetheme{Rochester}
}


\usepackage[english]{babel}
% or whatever

\usepackage[latin1]{inputenc}
% or whatever

\usepackage{times}
\usepackage[T1]{fontenc}
% Or whatever. Note that the encoding and the font should match. If T1
% does not look nice, try deleting the line with the fontenc.


\title[Synthetic Differential Geometry] % (optional, use only with long paper titles)
{A Brief Introduction to Synthetic Differential Geometry}

\author
{
  Gaspar P\'erez Scornik\\
  {\footnotesize Supervised by Xavier Gr\`acia Sabat\'e}
}

\institute[UPC] % (optional, but mostly needed)
{
  Universitat Polit\'ecnica de Catalunya\\
  Faculty of Mathematics and Statistics
}

\date[BSCT 2017] % (optional, should be abbreviation of conference name)
{Bachelor's Thesis, 2017}

% If you have a file called "university-logo-filename.xxx", where xxx
% is a graphic format that can be processed by latex or pdflatex,
% resp., then you can add a logo as follows:

\pgfdeclareimage[height=1.5em]{university-logo}{upc-logo.png}
\logo{\pgfuseimage{university-logo}}

% If you wish to uncover everything in a step-wise fashion, uncomment
% the following command: 

%\beamerdefaultoverlayspecification{<+->}

\begin{document}

\begin{frame}
  \titlepage
\end{frame}

\begin{frame}{Outline}
  \tableofcontents
  % You might wish to add the option [pausesections]
\end{frame}


% Structuring a talk is a difficult task and the following structure
% may not be suitable. Here are some rules that apply for this
% solution: 

% - Exactly two or three sections (other than the summary).
% - At *most* three subsections per section.
% - Talk about 30s to 2min per frame. So there should be between about
%   15 and 30 frames, all told.

% - A conference audience is likely to know very little of what you
%   are going to talk about. So *simplify*!
% - In a 20min talk, getting the main ideas across is hard
%   enough. Leave out details, even if it means being less precise than
%   you think necessary.
% - If you omit details that are vital to the proof/implementation,
%   just say so once. Everybody will be happy with that.

\section{Introductory theory}

\subsection{Single Variable Calculus}

\begin{frame}{The Derivative}

  \begin{itemize}
  \item
    ``The rate of change, for very small changes''
    \begin{equation*}
      f(x+h) = f(x) + f'(x)h + \mathcal O(h^2)
    \end{equation*}
  \end{itemize}
\end{frame}

\begin{frame}{Recreating the Derivative Axiomatically}

  \begin{itemize}
  \item 
    We begin with an object \( R \). Naturally, this is the ``synthetic version''
    of \( \R \).
  \end{itemize}
  \pause
  \begin{axiom}
    \( R \) is an algebra over \( \Q \).
  \end{axiom}
\end{frame}

\begin{frame}{Recreating the Derivative Axiomatically}{The Infinitesimals}

  \begin{itemize}
    \item Let \( D \) be the set
      \begin{equation*}
        D = \left\{ d\in R\mid d^2=0 \right\}.
      \end{equation*}
  \end{itemize}

  \pause

  \begin{klaxiom}
    For all functions
    \begin{equation*}
      f:D\to R,
    \end{equation*}
    there exist unique \( a,b\in R \) such that
    \begin{equation*}
      f(d) = a + bd
    \end{equation*}
    for all \( d\in D \).
    \begin{itemize}
      \item The coefficient \( a \) is easily seen to be \( f(0) \).
    \end{itemize}

  \end{klaxiom}
\end{frame}

\begin{frame}{Recreating the Derivative Axiomatically}
  \begin{itemize}
    \item The derivative can now be defined.
    \item Let \( f:R\to R \) be a function. Fix \( x \), define \( g:D\to R \) by
      \begin{equation*}
        g(d) = f(x+d).
      \end{equation*}
  \end{itemize}
\end{frame}

\begin{frame}{Recreating the Derivative Axiomatically}

  \begin{itemize}
    \item By the KL axiom, there exist unique \( a,b \) with
      \begin{equation*}
        g(d) = f(x+d) = a + bd
      \end{equation*}
    \item The first coefficient is \( g(0) = f(x) \).
  \end{itemize}
  \pause
  \begin{defn}
    The \alert{derivative} of \( f \), written \( f'(x) \), is the unique coefficient \( b \).
  \end{defn}

\end{frame}

\begin{frame}{Recreating the Derivative Axiomatically}
  \begin{itemize}
    \item Thus we have
      \begin{equation*}
        f(x+d) = f(x) + f'(x)d\quad \forall d\in D
      \end{equation*}
  \end{itemize}
\end{frame}

\begin{frame}{An Example}
  \begin{exmp}
    Consider \( f(x) = x^n \). Then
    \begin{align*}
      f(x+d) &= \binom{n}{0}x^n + \binom{n}{1}x^{n-1}d + K_2d^2 + \dots + K_nd^n  \\[2ex]
             &= x^n + \alert{nx^{n-1}}d
    \end{align*}
    Therefore \( f'(x) = nx^{n-1} \).
  \end{exmp}
\end{frame}

\subsection{A Primer on Toposes}

\begin{frame}{How is this possible?}
  \begin{itemize}
    \pause
    \item We have to leave the category of sets.
    \pause
    \item We have to weaken logic to only constructive logic (no LEM).
    \pause
    \item Toposes are the correct setting for SDG.
    \pause
    \item The objects in a topos behave like sets.
  \end{itemize}
\end{frame}

\section{Microlinear Objects}

\subsection{The Tangent Bundle}


\begin{frame}{Microlinear Objects}
  \begin{itemize}
    \item Define ``2-infinitesimals'':
      \begin{equation*}
        D(2) =  \left\{(d_1,d_2)\in D\times D \mid d_1d_2 = 0\right\}
      \end{equation*}
  \end{itemize}
\end{frame}

\begin{frame}{Microlinear Objects}

\begin{itemize}
  \item In particular, if \( M \) is microlinear then two maps
    \begin{equation*}
      f,g:D\to M
    \end{equation*}
    define a unique map
    \begin{equation*}
      h:D(2)\to M
    \end{equation*}
    such that 
    \begin{align*}
      h(d,0) &= f(d)\\
      h(0,d) &= g(d)
    \end{align*}
\end{itemize}

\end{frame}

\begin{frame}{Microlinear Objects}

\begin{itemize}
  \item To be thought of as ``manifolds''.
  \item We can define tangent vectors (and they will form a vector space).
\end{itemize}

\end{frame}

\begin{frame}{Tangent Vectors}

  \begin{itemize}
    \item Classically, they are equivalence classes of paths.
  \end{itemize}
  \pause
  \begin{defn}
    Let \( M \) be microlinear. A \alert{tangent vector} to \( M \) at \( p \) is a map
    \begin{equation*}
      t:D\to M
    \end{equation*}
    with \( t(0)=p \).
  \end{defn}
  \pause
  \begin{itemize}
    \item With that, the \alert{tangent space} at \( p \), \( T_pM \), is the set of all
      tangent vectors at p.
  \end{itemize}

\end{frame}

\begin{frame}{Tangent Vectors}
  \begin{itemize}
    \item The tangent space \( T_pM \) is an \( R \)-vector space.
      \begin{itemize}
        \item \( 0 \) = the constant map \( p \).
        \item \( \lambda t(d) := t(\lambda d)\)
      \end{itemize}
    \item For addition we need to use microlinearity.
  \end{itemize}
\end{frame}

\begin{frame}{Addition of Tangent Vectors}
  \begin{itemize}
    \item Microlinearity guarantees that for \( t_1, t_2 \) tangent vectors,
      there exists a unique
      \begin{equation*}
        s:D(2)\to M
      \end{equation*}
      such that
      \begin{align*}
        s(d,0) &= t_1(d)\\
        s(0,d) &= t_2(d)
      \end{align*}
    \item Define \( (t_1 + t_2)(d) = s(d,d) \)
  \end{itemize}
\end{frame}

%possible visualization frame

\begin{frame}{Addition of Tangent Vectors}{The Neutral Element}
  \begin{itemize}
    \item Recall we designated \( 0\equiv p\ \text{(ct.)} \)
    \item Let \( t \) be a t.v. The map \( s:D(2)\to M \) must satisfy
      \begin{align*}
        s(d,0) &= t(d,0)\\
        s(0,d) &= 0(d) = p
      \end{align*}
      so \( s \) must be
      \begin{equation*}
        s(d_1,d_2) = t(d_1),
      \end{equation*}
      and \( (t + 0)(d) = s(d,d) = t(d) \)
  \end{itemize}
\end{frame}

\begin{frame}{The Tangent Bundle}
  \begin{itemize}
    \item It is simply the collection of all tangent spaces, \( M^D \). We write \( TM \).
    \item We have a projection \( \pi:TM\to M \) given by
      \begin{equation*}
        t\mapsto t(0)
      \end{equation*}
  \end{itemize}
\end{frame}

\subsection{Vector Fields and Flows}

\begin{frame}{Vector Fields}
  \begin{itemize}
    \item They are sections of \( TM \); maps \( X:M\to TM \) such that \( (\pi\circ X)(p) = p \).
    \item The set of vector fields is denoted \( \vfld M \).
      \begin{itemize}
        \pause
        \item Each \( T_pM \) is a vector space, so by fiberwise operations so is \( \vfld M \).
        \pause
        \item \( \vfld M \) is also an \( M^R \)-module, by
          \begin{equation*}
            (fX)(p) = f(p)X(p)
          \end{equation*}
      \end{itemize}
  \end{itemize}
\end{frame}

\begin{frame}{Flows}
  \begin{itemize}
    \item Exponentials \( A^X \) have the property that
      \begin{equation*}
        B\to A^X 
      \end{equation*}
      is equivalent to
      \begin{equation*}
        X\times B \to A
      \end{equation*}\\
      \begin{equation*}
        \left( \tilde f(x,b) = f(b)(x) \right)
      \end{equation*}
  \end{itemize}
\end{frame}

\begin{frame}{Flows}
  \begin{itemize}
    \item Vector fields are maps \( X:M\to M^D \), therefore the same as maps
      \begin{equation*}
        X: D\times M \to M.
      \end{equation*}
      We write the same \( X \) because it's comfortable.
  \end{itemize}
\end{frame}

\begin{frame}
  \begin{proposition}
    Flows are additive.
  \end{proposition}
  \begin{proof}
    Let \( f,g:D(2) \) be defined by
    \begin{align*}
      f(d_1,d_2) &= X(d_1+d_2,p)\\
      g(d_1,d_2) &= X(X(d_1,p),d_2)
    \end{align*}
    we have
    \begin{align*}
      f(d,0) &= g(d,0)\\
      g(0,d) &= f(0,d)
    \end{align*}
    since \( M \) is microlinear, \( f=g \).
  \end{proof}
\end{frame}



\section*{Summary}

\begin{frame}{Summary}

  % Keep the summary *very short*.
  \begin{itemize}
  \item
    In SDG we define geometric objects directly.
  \item
    This requires working axiomatically.
  \item
    The right context for these needs is that of topos theory.
  \item 
    It's possible to recover the ``same'' constructs from classical geometry.
  \end{itemize}
  
  % The following outlook is optional.
  \vskip0pt plus.5fill
  \pause
  \begin{itemize}
    \item Of course, we need to construct a category which satisfies the axioms.
    \item These are \alert{models} of SDG.
  \end{itemize}
\end{frame}

\appendix
\section<presentation>*{\appendixname}
\subsection<presentation>*{For Further Reading}

\begin{frame}
  \frametitle<presentation>{For Further Reading}
    
  \begin{thebibliography}{10}
    
  \beamertemplatebookbibitems

  \bibitem{bun17}
    Marta~C. Bunge, Felipe Gago, and Ana~Mar\'ia San~Luis.
    \newblock Synthetic differential topology.
    \newblock Submitted manuscript, not yet published, 2017.
 
  \bibitem{kock06}
    Anders Kock.
    \newblock {\em Synthetic Differential Geometry}.
    \newblock Cambridge University Press, 2nd edition, 2006.
    
  \bibitem{kock10}
    Anders Kock.
    \newblock {\em Synthetic Geometry of Manifolds}.
    \newblock Cambridge University Press, 2010.
      
  \bibitem{lav96}
    Ren\'e Lavendhomme.
    \newblock {\em Basic Concepts of Synthetic Differential Geometry}.
    \newblock Springer Science+Business Media, B.V., 1996.

  \end{thebibliography}
\end{frame}

\end{document}
