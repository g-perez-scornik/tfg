\documentclass[11pt]{article}
\usepackage[a4paper,total={6.5in,8in}]{geometry}

\begin{document}
\begin{abstract}
  The goal of this monograph is to explore the basic axiomatic theory of Synthetic Differential Geometry. In case the reader is not familiar, Synthetic Differential Geometry (SDG) refers to a field of study which aims to put the study of smooth manifolds and geometry therein, in a topos-theoretic framework. Though the full depth of application and consequences of SDG require knowledge of topos-theory to comprehend, a large part of the theory can be appreciated with only some notions of basic category theory (as well as with a standard undergraduate mathematics syllabus). In this work we look at this part of SDG, called the ``axiomatic'' theory because it is indeed developed axiomatically. Specifically, under the axiomatic theory of SDG we look at differential calculus, then ``Manifolds'' (their analogue in SDG), Vector bundles (the tangent bundle as a particular case), vector fields (and their Lie algebras), and differential forms. At the end we dedicate a small section to looking at how SDG can be applied to recover results in classical differential geometry, that is, how one would translate from results in the axiomatic theory to ``real'' results pertaining to classical smooth manifolds.
\end{abstract}
\end{document}
