\chapter*{Introduction}

The title of this work may sound strange if one is unfamiliar with the topic. Of course, that is true of anything in life, but my experience in particular when stating the phrase ``Synthetic Differential Geometry'' to my mathematician colleagues (students and professors) is of being met immediately with a confused look, albeit one of intrigue. If they are a classmate of mine then they also will have likely not seen any category theory (what's that?) as it is not covered in the curriculum. I was in the same position when I did perchance upon the topic, and I was sufficiently intrigued so as to explore it myself and compile my findings into a short document that would hopefully serve as a light introduction for undergraduate students of mathematics.

What is \emph{synthetic} about synthetic differential geometry (SDG)? Quite simply, it means that it is an attempt to study, or at times simply describe the behavior of smooth manifolds (and smooth functions) from an axiomatic framework. To help explain this, we'll make an analogy with modern Euclidean planar geometry (\( \R^2 \) with a scalar product) vs. ``synthetic Euclidean geometry'', which is the name we give to the study of geometry as the Greeks did, from axioms concerning the elemental objects of geometry: points and lines, and relations between them. Euclides' five axioms were not completely rigorous by today's standards, but in that case we simply refer to any modern revisiting of axiomatic Euclidean geometry, such as the works of Hilbert or Tarski. The important point is the contrast with doing geometry with an analytic backdrop (\( \R^2 \)), from which we define lines and points set-theoretically. In axiomatic geometry one begins directly with lines and points and describes how they interact.

A highly similar approach is taken with synthetic differential geometry. Naturally, there is a considerable complexity gap between flat \( 2 \)-space, and general smooth manifolds. In accordance with the jump in complexity between the objects we wish to study there accompanies such a jump in the type of axioms we need. In particular, synthetic differential geometry begins by describing requirements for a certain \emph{category}. If the reader is not familiar, I would highly recommend they read one or both of \cite{sml71} and \cite{lei16}. One wouldn't have to read the entirety of either to understand most of what I write here, but I would certainly say that readers will find themselves completely comfortable with anything encountered here, having read the first five chapters of \cite{lei16}. From here on this knowledge is assumed, to the aim of avoiding highly verbose detours into purely category theoretic details.

After parsing what is meant (in informal terms) by ``synthetic'' differential geometry as we have done, the immediate reaction is to ask why we'd like to do that (other than for pure interest, which is not a bad answer). One practical answer is that it can give a rigorous foundation to certain ways that geometers would \emph{like} to think, such as the notion of ``infinitessimals'' (this is one of the first constructions in SDG here and in many sources), or higher order concepts such as the interpretation of vector fields as ``infinitessimal flows''. I am not a geometer, and my own motivation for studying SDG was in large part out of an interest in \emph{any} modern category-theoretic construction, but we could say that ``thinking synthetically'' in differential geometry was an idea that held much appeal, whatever I believed it meant at the time. I quote Dr. Anders Kock, quoting Sophus Lie in \cite{kock06}, should it spark the same intrigue it did in me some time ago:
\begin{quote}
  What is meant by “synthetic” reasoning? Of course, we do not know exactly what Lie meant, but the following is the way we would describe it: It deals with space forms in terms of their structure, i.e. the basic geometric and conceptual constructions that can be performed on them. Roughly, these constructions are the morphisms which constitute the base category in terms of which we work; the space forms themselves being objects of it.
\end{quote}
