\chapter{Introduction}

\section{Overview}

The thesis is organized as follows. Section one begins by laying down (a reduced version of) the axiomatic framework in which the rest of the text takes place. As mentioned, we can summarize that framework as to say ``there exists a category with these properties''. Again, the reader with a basic grasp of category theory will get along fine. Strictly speaking one should be familiar with the notion of a \emph{topos}, but we do not dedicate any large portion to the details, instead taking what ``high-level'' properties we need for the rest of our discourse. Those abstract (and convenient) properties of toposes will be detailed. The end result is that we end up with an object \( R \), the smooth line, which is more or less the synthetic version of \( \R \), the reals. Particularly, there is a subset of \( R \), called \( D \) throughout, which are to be the embodiment of ``first order infinitesimals''. In analysis we may sometimes like to think of ``infinitesimal increments'', but of course there exists no such a thing. We have suggestive notation such as \( f(x+h) = f(x) + f'(x)h + \mathcal O(h^2) \), and we may even think of the last term as ``negligible'' for small enough \( h \). Perhaps not in analysis proper anyway, but it is known as a useful technique for rough estimations in, say, numerics. With our object \( D \) this is made a reality. Elements of \( d \) are ``so small'' that their square is actually \( 0 \). Using \( D \) and a strong axiom about their behavior it is possible to revisit single and multivariable calculus, and such is done. By introducing further axioms one can speak of integration as well, but this is not covered here, and instead we refer the reader to \cite{lav96} for a discussion on this.

In the second section further (and final) axiomatics are given. In this way we mimic the style of many sources of SDG, which introduce successively stronger axioms until reaching the desired general statement. It is ``general'' in the sense that every axiom up till that point becomes a particular case. The reasons for going about it this way are twofold. One, because as an introductory text it's appropriate to ease the reader in. Though mentioned that category theory is required, the first axiom and the entire first section are very much palpable for any student of mathematics (this may sound confusing or contradictory but it becomes clear upon reading). Secondly, the nature of studying SDG itself often takes the same approach. Axioms are introduced as they are needed, and attention is given to which theories are possible within one set of axioms or another. In fact, this is indicated as the method for studying SDG in \cite{bun17}. In any case, the second section of this text arrives at the general axiomatics and employs them henceforth.

Until the mentioned point the main object of study is \( R \). As classical geometry moves from \( \R \) (or \( \R^n \)) to smooth manifolds, so do we explore ``manifolds'' in the third section. Instead of manifolds they are called \emph{microlinear objects}, because that is what they are. That is, ``microlinearity'' is a property of certain objects which essentially demands that they behave ``similar to \( R \)'', in a way that is made precise. And, as it will result, microlinearity is a sufficient condition for effectively carrying out the constructions we'd desire in differential geometry, namely the tangent bundle and vector fields. Furthermore, microlinear objects are sufficiently well-behaved under typical constructions such as the Cartesian product, yet they are necessarily more general than classical manifolds, for one because ``function sets'' \( N^M \) where \( M,N \) are microlinear, are too microlinear, something which cannot be said if we replaced ``microlinear'' with ``smooth manifold''.

Having said that microlinear objects are ``more general'' than manifolds, section four deals with ways in which they are very similar. Taking advantage of the properties of microlinear objects we can define tangent spaces in an eminently geometric way. In classical geometry tangent vectors can be defined as equivalence classes of curves passing through a point, parametrized by some interval. Here a tangent vector is literally a single infinitesimal curve, a map \( t:D\to M \), from the infinitesimal interval to the space in question. The collection of such curves (through a fixed point) exhibits a vector space structure, promptly named the tangent space at a point. The tangent bundle follows, and so do vector fields as sections of it. Finally, the set of vector fields on a space will be found to form a Lie algebra.

\section{Concluding remarks and further study}

The focus of this thesis was foremostly on taking a first look at the theory of SDG, as well as gaining practice with the application of category theory in modern mathematics (particularly as relates to geometric fields). It is interesting, to say the least, to find that one can start from purely axiomatic foundations, and arrive at the ``same'' familiar world of manifolds and smooth maps. Additionally, what is written here is only a small portion. In the literature one will find still more topics from classical geometry in their synthetic form. To name a few, these are differential forms - the exterior differential, integration of forms, leading up to De Rham theory \cite{lav96}; connections - covariant differentiation, torsion, curvature \cite{lav96}, metrics - Riemannian and pseudo-Riemannian metrics, the Levi-Civita connection \cite{kock10}; and Hamiltonian mechanics \cite{lav96},\cite{nish96}. The approach taken for those topics is of varying complexity, but altogether in line with the types of reasoning employed in this thesis.

Right alongside, or inseparable from the depth of geometric material that's possible to explore from here, are the subjects of category theory, categorical logic, sheaf theory, topos theory, etc. To say that those are worthy of multiple theses in their own right would be an criminal understatement. However, in researching the matter of this thesis, a solid ground has been laid on which to continue building. For any interested readers, a very recent publication in SDG can be found in \cite{bun17}, in which topos theory and categorical logic, in contrast to the present text, are at the forefront of discourse.

\section{Bibliographical remarks}

Much of the main matter in this thesis is drawn in combination from \cite{kock06} and \cite{lav96}, and between those two the former plays a bigger part, at least in direct citations or use of arguments. The two other main sources are \cite{bun17} and \cite{kock10}, which are cited only a few times but were useful as resources, seeing as the four books cover much of the same material in different ways, thus leading to a more complete understanding.

There are two books on category theory in the bibliography, \cite{lei16} and \cite{sml71}. Since the material is common to any introductory category theory text, and the style of this text is to take it as common knowledge, they are not cited anywhere directly. They have a place in the bibliography simply because they were the main foundation of an extensive previous study of category theory, and are recommended for any readers who would like to do the same.

In the short section on constructive logic, \cite{fra84} was consulted in order to complete a particular argument. Lastly, \cite{nish96}, though consulted extensively at an earlier phase of research, is mentioned only in the front matter as a reference for further study, in particular that of Hamiltonian mechanics. 

\section{Historical notes}

The majority of the literature points to SDG originating with William Lawvere, whose doctoral advisor was Samuel Eilenberg, co-founder of the theory of categories. In particular, the subject is first introduced in Chicago 1967, in his seminar \textit{Categorical Dynamics}. That is to say, \textit{dynamics}, from the point of view of category theory (as opposed to ``the dynamics of categories''). From there the subject gained collaborators, and the first book on SDG was published by Anders Kock in 1981 (here we consult the 2006 reprint of that book, \cite{kock06}). During or after that time many other works were published, by authors such as Belair, L.; Bunge, M.; Dubuc, E.J.; Koszul, J.-L.; Lavendhomme, R.; Moerdijk, I.; Penon, J.; Reyes, G.E.; and Wraith, G.E., among others. If one consults those sources they would find that, according to them, they are picking up on the works left by mathematicians before them in similar fields, notably Henri Cartan, Charles Ehresmann, Alexander Grothendieck, and Andr\'e Weil. 
