\chapter{Microlinearity} \label{sec:microlinearity}

\section{A simple example}

We begin by observing a few specific cases. Let
\begin{equation*}
   D\times D \to D 
\end{equation*}
be the multiplication map, i.e. \( (d_1,d_2)\mapsto d_1d_2 \). Nothing in our theory suggests that this map should be surjective, but for many purposes it suffices to have the following property:

\begin{proposition}
  If \( f_1,f_2:D \to R \) verify that, for each \( d_1,d_2\in D,\, f_1(d_1d_2) = f_2(d_1d_2)  \) then \( f_1=f_2 \).
\end{proposition}

That is, it suffices to know any function \( f:D\to R \) on products of elements of \( D \). This is referred to in \cite{kock06} as ``\emph{\( R \) thinks that the multiplication : \( D\times D\to D \) is surjective}''.

\begin{proof}
  Define \( g = f_1-f_2 \). By our hypothesis \( g \) is identically zero on all elements \( d_1d_2 \) with \( d_1,d_2\in D \). By the KL axiom (since the previous section this refers to axiom {//placeholder//}, but in this case axiom \ref{KL1} also suffices), there exist unique \( a,b\in R \) such that
  \begin{equation*}
    g(d)=a + bd\quad \forall d\in D 
  \end{equation*}
  On the one hand \( a=g(0)=g(0\cdot 0) = 0 \), so that \( g(d)=bd \). Thus, for all \( d_1,d_2\in D \) we have \( 0 = g(d_1d_2) = bd_1d_2 \), or in other words that \( bd_1d_2=0 \) for all \( d_1,d_2\in D \), which means that \( b=0 \).
\end{proof}

\section{A slightly more interesting example}

Consider two functions \( f,g:D\to X \) such that \( f(0)=g(0) \), where \( X \) is some set (object of \(  \sdgE) \). Here there is no reason to assume that this will define a map \( D(2)\to X \). It will be the case, however, if \( X \) is \( R \).

\begin{proposition}
  Let \( f,g:D\to R \) be functions such that \( f(0)=g(0) \). Then there exists a unique function \( h:D(2)\to R \) such that \( h(d,0) = f(d) \) and \( h(0,d)=g(d) \) for all \( d\in D \).
  \label{prop:d2}
\end{proposition}

\begin{proof}
  Let \( a_0 = g(0)=f(0) \). Then we have that for all \( d\in D \),
  \begin{align*}
    f(d) &= a_0 + a_1d \\
    g(d) &= a_0 + b_1d \\
  \end{align*}
  Thus, we simply define \( h \) by
  \begin{equation*}
    h(d_1,d_2) = a_0 + a_1d_1 + b_1d_2
  \end{equation*}
  This clearly satisfies the existence requirements. The Uniqueness is a consequence of the KL axiom. Let \( k \) be another function that satisfies the hypotheses. By the KL axiom applied to \( D(2) \),
  \begin{equation*}
    k(d_1,d_2) = \alpha_0 + \alpha_1d_1 + \alpha_2d_2
  \end{equation*}
  for unique \( \alpha_i \). By \( k(d,0)=f(d) \) we obtain \( \alpha_0=a_0 \) and \( \alpha_1 = a_1 \). Similarly \( \alpha_2=b_1 \). Since the \( \alpha_i \) are unique and determine \( k \), we get \( k=h \) (if preferred we could have concluded at the very definition of \( h \), since the KL axiom asserts that functions on \( D(2) \) are defined as such - the argument has simply been explicited a bit further here).

\end{proof}

\section{Perceived colimits}

The previous examples can be stated in diagrammatic form, which will then lend itself to a generalization in terms of (co)limits. We'll focus on the second example. Consider the following commutative diagram:

\begin{equation}
  \xymatrix{
    {\{0\}} \ar[d]_0 \ar[r]^0   & D \ar[d]^{i_2} \\
    D \ar[r]_{i_1}              & D(2)
  }
  \label{dg:pushout}
\end{equation}

Where \( i_1 \) and \( i_2 \) are the inclusion maps \( i_1(d)=(d,0),\, i_2(d)=(0,d) \) and \( 0 \) is the constant zero map. The previous observation that two maps \( f,g:D\to X \) do not have to define a map \( D(2)\to X \) is equivalent to saying that diagram \ref{dg:pushout} is not a pushout. But, if we apply the \( R^{(\farg)} \) functor to it, the result is:

\begin{equation*}
  \xymatrix{
    R^{\{0\}}       & R^D \ar[l]_{R^0}                           \\
    R^D \ar[u]^{R^0} & R^{D(2)} \ar[u]_{R^{i_2}} \ar[l]^{R^{i_2}} 
  }
  \label{dg:pullback}
\end{equation*}

\begin{proposition}
  Diagram \ref{dg:pullback} is a pullback.
  \label{prop:pullback}
\end{proposition}

%TODO:make sure of this and add to appendix
\begin{proof}
  First we will make the diagram a little neater. Firstly, \( R^{\{0\}} \) is isomorphic to \( R \). The map \( R^0 \) Takes \( f:D\to R \) and sends it to \( f\circ 0 \), which we simply identify with \( f(0) \) the same way we identified \( R^{\{0\}} \) with \( R \). Likewise, the map \( R^{i_1} \) sends \( g:D(2)\to R \) to \( g\circ i_1 \). Thus, for \ref{dg:pullback} to be a pullback, we'd need that the set \( R^D(2) \) be in bijection with the set of pairs \( f,g\in R^D \) such that \( f(0)=g(0) \). But this is precisely what we proved in proposition \ref{prop:d2}.
\end{proof}

How we refer to this phenomenon (or how it is often referred to) is \emph{``R perceives diagram \ref{dg:pushout} as a pushout}. This is of course generalizes, as we will see shortly. To do so we will give a few definitions first.

\begin{defn}
  Let \( \walg \) be the category of Weil algebras (a subcategory of the category of \( \Q \)-algebras \( \qalg \)). A \emph{good limit} of Weil algebras is a limit \( \left(L\stackrel{f_i}{\longrightarrow}D_i\right)_{i\in I} \) in \( \qalg \) which is also a limit in \( \walg \). That is, the morphisms in the diagram defining \( L \) must be Weil algebra homomorphisms, and so must the projections \( f_i \).
\end{defn}

\begin{defn}
  A diagram in \( \sdgE \) is said to be a (finite) \emph{quasi colimit} if it is the image under the functor
  \begin{equation*}
    \spec_R(\farg):\walg\to\sdgE
  \end{equation*}
  of a good finite colimit in \( \walg \).
\end{defn}

\begin{defn}
  An object \( X \) is said to perceive a quasi colimit as a colimit if the functor
  \begin{equation*}
    X^{(\farg)}:\sdgE\to\sdgE
  \end{equation*}
  takes said quasi colimit into a limit (of \( \sdgE \)).
\end{defn}

\begin{exmp}
  The following diagram is a good colimit of Weil algbras (it is a pushout in \( \qalg \), and the morphisms are Weil algebra homomorphisms).
  \begin{equation}
    \xymatrix{
      {\Q}                   & {\Q[X]/(X^2)} \ar[l]^0 \\
      {\Q[X]/(X^2)} \ar[u]^0 & {\Q[X,Y]/(X^2,Y^2,XY)} \ar[u]^{p_2} \ar[l]^{p_1}
    }
    \label{dg:goodlimex}
  \end{equation}
  where \( p_1(X)=X, p_1(Y)=0, p_2(X)=0 \), and \( p_2(Y)=X \). Via the \( \spec_R(\farg) \) functor this is taken to
  \begin{equation*}
    \xymatrix{
      {\{0\}} \ar[d]_0 \ar[r]^0 & D \ar[d]^{i_2} \\
      D \ar[r]_{i_1}            & D(2)
    } 
  \end{equation*}
  which is the same as diagram \ref{dg:pushout}. Thus, since diagram \ref{dg:goodlimex} is an example of a good finite limit, diagram \ref{dg:pushout} is a finite quasi colimit. Furthermore, \( R \) perceives \ref{dg:pushout} as a limit, as observed in proposition \ref{prop:pullback}.
\end{exmp}

%TODO: add other places where it is useful, if they arise
For plenty more examples, one should look at \cite{lav96}. The objective of these definitions are to state the upcoming proposition, which generalizes the previous observation that \emph{``R perceives\dots etc''}. The result will be of use in section {//placeholder//} as a smoothness condition.

%TODO: converse
\begin{proposition}
  \( R \) perceives finite quasi colimits as colimits. Conversely, if \( R \) perceives a diagram as a colimit, it is a quasi colimit. 
  \label{prop:Rperc}
\end{proposition}

We'll need a simple result previously, in the form of the following

%TODO:proof
\begin{lemma}
  Let 
  \begin{equation*}
   \left(W_i\stackrel{p_i}{\longleftarrow}W_L\right)_{i\in I}
  \end{equation*}
  be any limit of Weil algebras. Then the diagram
  \begin{equation*}
    \left(R[W_i]\stackrel{p_i}{\longleftarrow}R[W_L]\right)_{i\in I}
  \end{equation*}
  is a limit of \( R \)-algebras.  
  \label{lm:Rbracket}
\end{lemma}

With that, we can prove the proposition.

\begin{proof}[Proof of proposition \ref{prop:Rperc}]
  Let \( \left(L\stackrel{f_i}{\longleftarrow}D_i\right)_{i\in I} \) be a finite quasi colimit. That is, we may write the diagram as
  \begin{equation}
    \left(\spec_R(W_L)\stackrel{\spec_R(p_i)}{\longleftarrow}\spec_R(W_i)\right)_{i\in I}
    \label{dg:fquasicolt}
  \end{equation}
  where
  \begin{equation*}
   \left(W_i\stackrel{p_i}{\longleftarrow}W_L\right)_{i\in I}
  \end{equation*}
  is some good finite limit of Weil algebras. We need to prove that diagram \ref{dg:fquasicolt} is taken to a limit by the functor \( R^{(\farg)} \). Under this functor, the resulting diagram is
  \begin{equation}
    \left(R^{\spec_R(W_L)}\stackrel{R^{\spec_R(p_i)}}{\longrightarrow}R^{\spec_R(W_i)}\right)_{i\in I}
    \label{dg:Rfqc}
  \end{equation}
  Now we invoke the general KL axiom, which says that this diagram is isomorphic to
  \begin{equation*}
    \left(R[W_i]\stackrel{p_i}{\longleftarrow}R[W_L]\right)_{i\in I}
  \end{equation*}
  By lemma \ref{lm:Rbracket}, this is a limit, since by hypothesis
  \begin{equation*}
   \left(W_i\stackrel{p_i}{\longleftarrow}W_L\right)_{i\in I}
  \end{equation*} 
  is a limit of Weil algebras.
\end{proof}

\section{Microlinear objects}

The behavior of \( R \) studied above will be a fundamental requirement as we move onto studying other objects of synthetic differential geometry. We call these objects \emph{microlinear}, and it can be said that they are a sort of generalized ``manifold'' in the context of SDG. Let us state this precisely:

\begin{defn}
  An object \( M \) is said to be \emph{microlinear} if it perceives finite quasi colimits as limits.
\end{defn}

Thus, obviously \( R \) is microlinear. We'll make two observations.

\begin{proposition}
  If \( M \) is a microlinear object and \( X \) is any other object, then \( M^X \) is also microlinear.
  \label{prop:expmicro}
\end{proposition}

\begin{proposition}
  If
  \begin{equation*}
    \left( M\stackrel{f_\alpha}{\longrightarrow} M_\alpha \right)_{\alpha\in A}
  \end{equation*}
  is a limit, with \( M_\alpha \) microlinear for every \( \alpha\in A \), then \( M \) is microlinear.
  \label{prop:limmicro}
\end{proposition}

