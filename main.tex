% B.Sc. Thesis - Synthetic Differential Geometry
% Gaspar Pérez Scornik <gasparperez2010@gmail.com>
%

\documentclass[12pt,twoside]{amsbook}

\usepackage[utf8]{inputenc}
\usepackage{cite}
\usepackage{amsmath}
\usepackage{amsfonts}
\usepackage[all]{xy}
\xymatrixrowsep{8ex}
\xymatrixcolsep{8ex}

\usepackage{amsthm}
\newtheorem{axiom}{Axiom}[section]
\setcounter{axiom}{-1}
\newtheorem{klaxiom}{KL Axiom}

\newtheorem{proposition}{Proposition}[chapter]
\newtheorem{lemma}[proposition]{Lemma}

\theoremstyle{definition}
\newtheorem{defn}[proposition]{Definition}
\newtheorem{exmp}[proposition]{Example}

\numberwithin{section}{chapter}
\numberwithin{equation}{chapter}

\mathchardef\mhyph="2D
\newcommand{\abs}[1]{\lvert #1 \rvert} %Absolute value
\newcommand{\ddx}[2]{\frac{\partial #1}{\partial #2}} %partial derivative
\newcommand{\ddxk}[4]{\frac{\partial^{#1} #2}{\partial #3 \cdots \partial #4}} %k-th partial derivative
\newcommand{\ddxII}[3]{\frac{\partial^{2} #1}{\partial #2 \partial #3}} %2nd partial derivative
\newcommand{\farg}{-} %empty argument in function
\newcommand{\quot}[2]{#1/#2} %quotient of groups, modules, etc.
\newcommand{\sdgE}{\mathcal{E}}
\newcommand{\Q}{\mathbb{Q}}
\newcommand{\vfld}[1]{\mathfrak X(#1)}
%TODO: make this look better
\newcommand{\qalg}{\Q\mhyph\mathbf{Alg}}
\newcommand{\walg}{\mathbf W}
\newcommand{\mlin}{\mathbf{Mic}}
\newcommand{\vbun}{\mathbf{Fib}}
\renewcommand{\vec}{\underline} %markup for vectors
\DeclareMathOperator{\spn}{span}
\DeclareMathOperator{\spec}{Spec}
\DeclareMathOperator{\rspec}{Spec_R}
\DeclareMathOperator{\id}{id}
\DeclareMathOperator{\aut}{Aut}


\DeclareRobustCommand{\gobblefour}[4]{}
\newcommand*{\SkipTocEntry}{\addtocontents{toc}{\gobblefour}}

\begin{document}

\includepdf{portada.pdf}

\thispagestyle{empty}

\begin{titlepage}
	\centering
	{\scshape\Large Universitat Polit\`ecnica de Catalunya\par}
	\vspace{0.5cm}
	{\scshape\Large Faculty of Mathematics and Statistics\par}
	\vspace{0.5cm}
	{\scshape\Large Degree in Mathematics\par}
	\vspace{1.5cm}
	{\huge\bfseries A Brief Introduction to Synthetic Differential Geometry\par}
	\vspace{2cm}
	{\Large\bfseries Gaspar P\'erez Scornik\par}
	\vspace{0.3cm}
	{\texttt gasparperez2010@gmail.com\par}
	\vfill
	Bachelor's Thesis\par
	supervised by\par
	\vspace{0.3cm}
	Xavier Gr\`acia Sabat\'e\par
  Department of Mathematics

	\vfill

	{\large September, 2017}
\end{titlepage}

\frontmatter

\setcounter{page}{2}

\begin{center}
  2010 {\itshape Mathematics Subject Classification:} 51K10, 03G30, 18F15, 58A05\par
  \vspace{0.3cm}
  {\itshape Key words:} synthetic differential geometry, smooth toposes, smooth manifolds, intuitionistic logic\par
\end{center}

\vspace{0.5cm}

\noindent{\scshape Abstract}. The goal of this thesis is to explore the basic axiomatic theory of Synthetic Differential Geometry (SDG). This field aims to put the study of smooth manifolds, and geometry therein, in a topos-theoretic framework. Though the full depth of application and consequences of SDG
require knowledge of topos theory to comprehend, a large part of the theory can be appreciated
with only some notions of basic category theory (as well as with a standard undergraduate
mathematics syllabus). In this work we look at this part of SDG, called the “axiomatic” theory
because it is indeed developed axiomatically. Specifically, under the axiomatic theory of SDG
we look at differential calculus, then “manifolds” (their analogue in SDG), vector bundles (the
tangent bundle as a particular case), and vector fields (and Lie algebras thereof).\par
\vspace{0.2cm}

\par\noindent\rule[0.2cm]{\textwidth}{0.4pt}

\noindent{\scshape Resum}. L'objectiu d'aquest treball \'es el d'explorar la teoria axiom\`atica b\`asica de l'anomenada \emph{Geometria Diferencial Sint\`etica} (GDS). La GDS es refereix a un camp que t\'e l'objectiu d'enquadrar l'estudi de les varietats diferenciables i la seva geometria dins el marc de la teoria de topos. Malgrat que l'apreci complet de les aplicacions i conseq\"u\`encies d'aquesta teoria requereixen con\`eixer la teoria de topos, una gran part de la teoria es pot tractar amb nom\'es unes nocions b\`asiques de teoria de categories (juntament amb una educaci\'o en matem\`atiques a nivell de grau). En aquest treball visitem aquesta part de la teoria, que s'anomena ``axiom\`atica'' perqu\'e efectivament ho \'es. En particular sota la teoria axiom\`atica de la GDS tractem el c\`alcul diferencial, varietats (el seu an\`aleg), fibrats vectorials (el fibrat tangent en especial), i finalment camps vectorials (amb la seva estructura d'\`algebra de Lie).

\clearpage
\SkipTocEntry\chapter*{Acknowledgments}
I would like to thank my advisor Xavier Gràcia for his counsel and direction, the BGSM for hosting a wonderful course on category theory, and my friends and family.

This thesis was typeset using the \LaTeX\ typesetting system originally developed by Leslie Lamport, based on \TeX\ created by Donald Knuth. Diagrams and figures were drawn using the \texttt{XY-pic} package.

\tableofcontents
\chapter*{Preface}

The title of this work may sound strange if one is unfamiliar with the topic. Of course, that is true of anything in life, but my experience in particular when stating the phrase ``Synthetic Differential Geometry'' to my mathematician colleagues (students and professors) is of being met immediately with a confused look, albeit one of intrigue. If they are a classmate of mine then they also will have likely not seen any category theory (what's that?) as it is not covered in the curriculum. I was in the same position when I did perchance upon the topic, and I was sufficiently intrigued so as to explore it myself and compile my findings into a short document that would hopefully serve as a light introduction for undergraduate students of mathematics.

What is \emph{synthetic} about synthetic differential geometry (SDG)? Quite simply, it means that it is an attempt to study, or at times simply describe the behavior of smooth manifolds (and smooth functions) from an axiomatic framework. To help explain this, we'll make an analogy with modern Euclidean planar geometry (\( \R^2 \) with a scalar product) vs.\ ``synthetic Euclidean geometry'', which is the name we give to the study of geometry as the Greeks did, from axioms concerning the elemental objects of geometry: points and lines, and relations between them. Euclides' five axioms were not completely rigorous by today's standards, but in that case we simply refer to any modern revisiting of axiomatic Euclidean geometry, such as the works of Hilbert or Tarski. The important point is the contrast with doing geometry with an analytic backdrop (\( \R^2 \)), from which we define lines and points set-theoretically. In axiomatic geometry one begins directly with lines and points and describes how they interact.

A highly similar approach is taken with synthetic differential geometry. Naturally, there is a considerable complexity gap between flat \( 2 \)-space, and general smooth manifolds. In accordance with the jump in complexity between the objects we wish to study there accompanies such a jump in the type of axioms we need. In particular, synthetic differential geometry begins by describing requirements for a certain \emph{category}. If the reader is not familiar, I would highly recommend they read one or both of \cite{sml71} and \cite{lei16}. One wouldn't have to read the entirety of either to understand most of what I write here, but I would certainly say that readers will find themselves completely comfortable with anything encountered here, having read the first five chapters of \cite{lei16}. From here on this knowledge is assumed, to the aim of avoiding highly verbose detours into purely category theoretic details.

After parsing what is meant (in informal terms) by ``synthetic'' differential geometry as we have done, the immediate reaction is to ask why we'd like to do that (other than for pure interest, which is not a bad answer). One practical answer is that it can give a rigorous foundation to certain ways that geometers would \emph{like} to think, such as the notion of ``infinitesimals'' (this is one of the first constructions in SDG here and in many sources), or higher order concepts such as the interpretation of vector fields as ``infinitesimal flows''. I am not a geometer, and my own motivation for studying SDG was in large part out of an interest in \emph{any} modern category-theoretic construction, but we could say that ``thinking synthetically'' in differential geometry was an idea that held much appeal, whatever I believed it meant at the time. I quote Anders Kock, quoting Sophus Lie in \cite{kock06}, should it spark the same intrigue it did in me some time ago:
\begin{quote}
  What is meant by “synthetic” reasoning? Of course, we do not know exactly what Lie meant, but the following is the way we would describe it: It deals with space forms in terms of their structure, i.e.\ the basic geometric and conceptual constructions that can be performed on them. Roughly, these constructions are the morphisms which constitute the base category in terms of which we work; the space forms themselves being objects of it.
\end{quote}


\mainmatter

\chapter*{Introduction}

The title of this work may sound strange if one is unfamiliar with the topic. Of course, that is true of anything in life, but my experience in particular when stating the phrase ``Synthetic Differential Geometry'' to my mathematician colleagues (students and professors) is of being met immediately with a confused look, albeit one of intrigue. If they are a classmate of mine then they also will have likely not seen any category theory (what's that?) as it is not covered in the curriculum. I was in the same position when I did perchance upon the topic, and I was sufficiently intrigued so as to explore it myself and compile my findings into a short document that would hopefully serve as a light introduction for undergraduate students of mathematics.

What is \emph{synthetic} about synthetic differential geometry (SDG)? Quite simply, it means that it is an attempt to study, or at times simply describe the behavior of smooth manifolds (and smooth functions) from an axiomatic framework. To help explain this, we'll make an analogy with modern Euclidean planar geometry (\( \R^2 \) with a scalar product) vs. ``synthetic Euclidean geometry'', which is the name we give to the study of geometry as the Greeks did, from axioms concerning the elemental objects of geometry: points and lines, and relations between them. Euclides' five axioms were not completely rigorous by today's standards, but in that case we simply refer to any modern revisiting of axiomatic Euclidean geometry, such as the works of Hilbert or Tarski. The important point is the contrast with doing geometry with an analytic backdrop (\( \R^2 \)), from which we define lines and points set-theoretically. In axiomatic geometry one begins directly with lines and points and describes how they interact.

A highly similar approach is taken with synthetic differential geometry. Naturally, there is a considerable complexity gap between flat \( 2 \)-space, and general smooth manifolds. In accordance with the jump in complexity between the objects we wish to study there accompanies such a jump in the type of axioms we need. In particular, synthetic differential geometry begins by describing requirements for a certain \emph{category}. If the reader is not familiar, I would highly recommend they read one or both of \cite{sml71} and \cite{lei16}. One wouldn't have to read the entirety of either to understand most of what I write here, but I would certainly say that readers will find themselves completely comfortable with anything encountered here, having read the first five chapters of \cite{lei16}. From here on this knowledge is assumed, to the aim of avoiding highly verbose detours into purely category theoretic details.

After parsing what is meant (in informal terms) by ``synthetic'' differential geometry as we have done, the immediate reaction is to ask why we'd like to do that (other than for pure interest, which is not a bad answer). One practical answer is that it can give a rigorous foundation to certain ways that geometers would \emph{like} to think, such as the notion of ``infinitesimals'' (this is one of the first constructions in SDG here and in many sources), or higher order concepts such as the interpretation of vector fields as ``infinitesimal flows''. I am not a geometer, and my own motivation for studying SDG was in large part out of an interest in \emph{any} modern category-theoretic construction, but we could say that ``thinking synthetically'' in differential geometry was an idea that held much appeal, whatever I believed it meant at the time. I quote Dr. Anders Kock, quoting Sophus Lie in \cite{kock06}, should it spark the same intrigue it did in me some time ago:
\begin{quote}
  What is meant by “synthetic” reasoning? Of course, we do not know exactly what Lie meant, but the following is the way we would describe it: It deals with space forms in terms of their structure, i.e. the basic geometric and conceptual constructions that can be performed on them. Roughly, these constructions are the morphisms which constitute the base category in terms of which we work; the space forms themselves being objects of it.
\end{quote}

\chapter{The basic theory} \label{sec:basictheory}

The axiomatic theory of Synthetic Differential Geometry begins by assuming a certain topos \( \sdgE \) exists. A topos is a category with certain properties which are meant to abstract the way in which the category of sets behaves. In fact, most naive set theoretic and logical notions have meaningful interpretation in terms of objects and morphisms of a topos. Thus, from this point on we will mainly omit the fact that we are working in a topos, and use standard set theoretic notation, giving only brief reminders now and then. 

\section{Axiomatics}

We begin with an object \( R \). We can think of \( R \) as an extension of the notion of the standard continuum, \( \mathbb{R} \). Thus \( R \) should be a commutative ring with unit. However, due to an axiom that we will shortly introduce, we cannot ask that \( R \) be a field. Later on, we will need to strengthen this requirement to, for example ``2:=1+1 is invertible''. But later on we may need that ``3 is invertible''. Instead of introducing these small axioms successively, we will take care of them succinctly with our:

\begin{axiom}
  \label{ax0}
  \( R \) is a \( \Q \)-algebra.
\end{axiom}

With that out of the way, we proceed to introduce a central object of study, the ``infinitesimals''. These come in the form of nilpotent elements of \( R \). To be precise, let \( D = \{d\in R \mid d^2=0\} \). The (first version of the) defining axiom of SDG concerns \( D \), and as in all the literature we refer to it as the Kock-Lawvere axiom.

\begin{klaxiom}
  \label{KL1}
  For all \( f:D\to R \) there exist unique \( a,b\in R \) such that \( f(d) = a + bd\,\forall d\in D \)
\end{klaxiom}

Note that, in particular, \( a=f(0) \). Of course, this is a strong requirement. So strong in fact, that we will have to weaken our logic in order for the theory to be remotely interesting. This is because of the following proposition.

\begin{proposition}
  \label{0prop}
  \( R=\{0\} \)
\end{proposition}

\begin{proof}
  Let \( f:D\to R \) be defined by
  \begin{equation*}
    f(d) =
    \left\{ 
      \begin{aligned}
	1,& & d \neq 0 \\
	0,& & d = 0
      \end{aligned}
    \right.
  \end{equation*}
By the KL axiom \ref{KL1}, there exist unique \( a,b\in R \) such that \( f(d) = a + bd \) for all \( d \). Since \( a =  f(0) = 0 \), there exists \( b\in R \) such that \( f(d) = bd \) for all \( d \). Thus for non-zero \( d \) we have \( 1 = bd \). Squaring both sides yields \( 1 = 0 \), and since \( R \) is a ring this concludes the proof.
\end{proof}

However, the above proof relies on the ``fact'' that, for any \( x\in R \) we have either \( x = 0 \) or \( x\neq0 \). Indeed the well-definedness of the function \( f \) depends on this assertion. This is true in classical logic, as it is a simple application of the \textbf{Law of the Excluded Middle} (LEM):
\begin{equation*}
  \forall p\, p\vee \neg p
\end{equation*}
%TODO: comment about internal logic of topos is intuitionistic
To carry on with our theory, we see that we have to reject LEM. The above proof then fails, but of course just because of that we cannot conclude that the theory functions properly. To do so would by all means require exhibiting the category \( \sdgE \) explicitly. For now we will continue ignoring this, and develop the naive theory, knowing that we have to abstain from assuming LEM.

Without LEM, (\ref{0prop}) of course still contradicts (because we do want R to be a \( \Q \)-algebra) the statement
\begin{equation}
  \forall d\in D\,(d=0) \vee \neg (d=0)
  \label{eq:not0}
\end{equation}
In other words, its negation is true. One may be inclined to ``simplify'' the negation statement, leading to the apparent absurdity that
\begin{equation*}
  \exists d\in D\, \neg(d=0) \wedge \neg\neg (d=0)
\end{equation*}
Note that we haven't eliminated the double negation, since \( \neg\neg p\rightarrow p \) is equivalent to LEM. Still, the above would be a contradiction, since \( p\wedge \neg p \) is still false in intuitionistic logic. But, the first equivalence we used on the universally quantified statement (\ref{eq:not0}) also fails to be valid in intuitionistic logic (see \cite[p. 248]{fra84} for example).

%TODO: find a better proof of this
So what are the elements of \( D \)? For one they are not all zero, for if they were then every function \( d\mapsto ad, a\in R \) would be equal to the constant \( 0 \), but since each coefficient \( a \) is unique, we again reach the absurdity that every \( a\in R \) is the same (equal to zero). At the same time the negation of (\ref{0prop}) implies that not all \( d \) that aren't zero satisfy \( \neg(d=0) \). The consequence is the simple statement that
\begin{equation*}
  \exists d\in D \neg\neg(d=0)
\end{equation*}

The infinitesimals \( D \) are thus neither zero nor nonzero. This is a contradictory statement in classical logic, but it is precisely of the type of phrases that we should expect to appear if we weaken our logic to intuitionistic logic.

%TODO: cancellation of universally quantified d
\section{Elementary Calculus}

With our basic axiom scheme in place we can begin revisiting the classical notion of a derivative in this new context. We begin, as expected, with a function \( f:R\to R \). Fix \( x\in R \) and define a new function \( g:D\to R \) by
\begin{equation*}
  d\mapsto f(x+d)
\end{equation*}
By the KL axiom \ref{KL1} there are unique \( a,b\in R \) such that
\begin{equation*}
  \forall d\in D\, f(x+d) = f(x)+bd
\end{equation*}
(since \( g(0) = f(x) \)). This naturally leads to the following
\begin{defn}
  Let \( f:R\to R \). The \emph{derivative} of \( f \) at \( x \), denoted \( f'(x) \), is the unique \( b\in R \) such that \( f(x+d)=f(x)+bd\,\forall d\in D \).
  \label{def:1varder}
\end{defn}

Clearly this also allows us to define the derivative \emph{function} \( f':R\to R \). For instance, let us calculate the derivative of a simple polynomial function. Let \( f:x\mapsto x^2+x \) then
\begin{align*}
  f(x+d) & =  x^2+2xd + d^2 + x + d \\ 
         & =  x^2 + x + (2x+1)d
\end{align*}
by which \( f'(x)=2x+1 \). So far our ``formal'' calculus agrees with standard calculus. In fact, we have the following proposition:

\begin{proposition}
  Let \( f,g:R\to R \) and let \(\alpha\in R.\) The following hold:
  \begin{enumerate}
    \item \((f+g)' = f'+g'\)
    \item \((\alpha f)' = \alpha f'\)
    \item \((fg)' = f'g + fg'\)
  \end{enumerate}
\end{proposition}
\begin{proof}
  The proof is as simple as one could suspect, here we only prove the third identity as an example. We have:
  \begin{align*}
    f(x+d)g(x+d) & = (f(x)+f'(x)d)(g(x)+g'(x)d) \\
                 & = f(x)g(x) + (f'(x)g(x)+f(x)g'(x))d + f'(x)g'(x)d^2 \\
		 & = f(x)g(x) + (f'(x)g(x)+f(x)g'(x))d  
    \label{eq:prodrule}
  \end{align*}
  At a given \( x\in R \), this clearly holds for all \( d\in D \), giving us the result.
\end{proof}

Furthermore we have the Chain Rule:
\begin{proposition}
  Let \( f,g:R\to R \) be two functions. Then the derivative of the composite \( g\circ f \) is
  \begin{equation*}
    (f\circ g) = (f'\circ g)\cdot g'
  \end{equation*}
\end{proposition}
\begin{proof}
  Let \( x\in R \). Then
  \begin{equation*}
   (f\circ g)(x+d) = f(g(x)) + (f\circ g)'(x)d
  \end{equation*}
  by definition. Meanwhile, if we expand \( g(x+d) \) first:
  \begin{align*}
    (f\circ g)(x+d) & = f(g(x+d)) \\
                    & = f(g(x)+g'(x)d) \\
  \end{align*}
  Observe that \( g'(x)d\in D \), since \( (g'(x)d)^2 = g'(x)^2d^2 = 0 \). Now, \( f'(g(x)) \) is the unique coefficient satisfying
  \begin{equation*}
    f(g(x)+g'(x)d) = f(g(x)) + f'(g(x))g'(x)d
  \end{equation*}
\end{proof}

The reader may notice that these statements seem lazily worded. Normally, we would include in the hypotheses something like ``\textit{Let \( f,g:R\to R \) be} differentiable \textit{functions. Then} (some other function) \textit{is differentiable, and \dots etc.}'' We are not in fact forgetting hypotheses. The Kock-Lawvere Axiom \ref{KL1} actually implies that \textit{every} function from \( R \) to \( R \) is differentiable. This is easy to verify by noting that in the definition of the derivative (\ref{def:1varder}), no additional assumptions are made other than \( f \) being a function. As a direct consequence, every function from \( R \) to \( R \) is infinitely differentiable.

This leads us to ask if we also have Taylor expansions of functions. In classical calculus, a Taylor expansion of second order involves a second degree polynomial, and ``terms of order 3''. In our context, these are simply nilcube elements, i.e. \( \delta\in R : \delta^3 = 0 \). An example of such a nilcube is any \( d_1+d_2 \) where \( d_1,d_2 \) are both in \( D \). In effect,
\begin{equation*}
  (d_1+d_2)^3 = d_1^3 + d_2^3 + 3d_1^2d_2 + 3d_1d_2^2 = 0
\end{equation*}

Furthermore, if \( f:R\to R\) is a function, 
\begin{align*}
  f(x+d_1+d_2) & = f(x+d_1)+f'(x+d_1)d_2 \\
               & = f(x) + f'(x)d_1 + f'(x)d_2 + f''(x)d_1d_2
\end{align*}
and meanwhile,
\begin{align*}
  (d_1+d_2)^2 & = 2d_1d_2
\end{align*}
Joining the two together yields:
\begin{equation*}
  f(x+d_1+d_2) = f(x) + f'(x)(d_1+d_2) + \frac{f''(x)}{2}(d_1+d_2)^2
\end{equation*}

What we have just proven is the following proposition.
\begin{proposition}
  If \( f:R\to R \) is a function and \( d_1,d_2\in D \), then \( \delta = d_1+d_2 \) satisfies
  \begin{equation*}
    f(x+\delta) = f(x) + f'(x)\delta + \frac{f''(x)}{2}\delta^2
  \end{equation*}

\end{proposition}

This justifies the earlier heuristic of considering nilcubes as candidates for second order Taylor expansions. Unfortunately, there is no way to assert as of now that \textit{any} nilcube must be of the form \( d_1+d_2 \) with \( d_1,d_2\in D \). It is a partial result. In section {/placeholder1/} we will introduce Taylor expansions for every nilpotent element by way of an axiom that generalizes the KL axiom \ref{KL1}. Still, it serves as a nice exercise to prove the following, more general, Taylor formula.

\begin{proposition}
  Let \( f:R\to R \) be a function. Then for all \( n\in\mathbb N \) and \( \delta = d_1+\cdots+d_n \)
  \begin{equation*}
    f(x+\delta) = f(x) + f'(x)\delta + \frac{f''(x)}{2!}\delta^2 + \cdots + \frac{f^{(n)}}{n!}\delta^n
  \end{equation*}
  where \( x\in R \), \( d_1,\dots,d_n\in D \)
\end{proposition}

\begin{proof}
  To begin with, let us look at the powers of \( \delta \), that is, the usual multinomial formula:
  \begin{equation*}
    \delta^k = (d_1+\cdots+d_n)^k = \sum_{i_1+\cdots+i_n=k}\left( k\atop i_1,\dots,i_n \right)\prod_{r=1}^{n}d_r^{i_r}
  \end{equation*}
  Recall that the \( d_r \) are in \( D \), so only the only non-zero summands will be those with each \( d_r \) raised to either 1 or 0. That is, choosing a subset of the \( i_r \) to be 1, and the rest 0. In that case the multinomial coefficients are simply \( k! \), giving us
  \begin{equation*}
    \delta^k = \sum_{I\subset \{1,\dots,n\}\atop \abs{I}=k}k!\prod_{t\in I}d_t
  \end{equation*}
  in other words, the \( k \)-\textit{th} elementary symmetric polynomial of \( n \) variables (multiplied by \( k! \)). Denote this by \( e_k(X_1,\dots,X_n) \), and convene \( e_0 \equiv 1 \) and \( e_k(X_1,\dots,X_n) = 0 \) if \( k>n \). We proceed inductively:
  % [TODO] Fix this equation
  \begin{align}
    \begin{split}
      f(x+\delta) & = f(x+d_1+\cdots+d_n) \\[10pt]
                  & = f(x+d_1+\cdots+d_{n-1})+f'(x+d_1+\cdots+d_{n-1})d_n \\[10pt]
		  & = \sum_{i=0}^{n-1}\frac{f^{(i)}(x)}{i!}i!e_i(d_1,\dots,d_{n-1}) + d_n\sum_{i=0}^{n-1}\frac{f^{(i+1)}(x)}{i!}i!e_i(d_1,\dots,d_{n-1}) \\[10pt]
                  & = \sum_{i=0}^{n-1}f^{(i)}(x)\left( e_i(d_1,\dots,d_{n-1})+d_ke_{i-1}(d_1,\dots,d_{n-1})\right) + d_kf^{(n)}(x)e_{k-1}(d_1,\dots,d_{k-1})
    \end{split}
    \label{eq:ntaylor}
  \end{align}
It's now useful to use the following recursion that elementary symmetric polynomials satisfy:
\begin{equation*}
  e_k(X_1,\dots,X_n) = e_{k}(X_1,\dots,X_{n-1}) + X_{n}e_{k-1}(X_1,\dots,X_{n-1})
\end{equation*}
Finally, observe that \( e_{k-1}(d_1,\dots,d_{k-1}) \) is just \( d_1d_2\cdots d_{k-1} \), meaning that \( d_ke_{k-1}(d_1,\dots,d_{k-1}) = e_k(d_1,\dots, d_{k-1},d_k) \). With these two identities, \ref{eq:ntaylor} becomes:
\begin{equation*}
  \sum_{i=0}^{n}f^{(i)}(x)e_i(d_1,\dots,d_k)= \sum_{i=0}^n \frac{f^{(i)}(x)}{i!}\delta^j
\end{equation*}
\end{proof}

We'd do well to note that, for \( n\geq k+1 \), \( \delta^n \) is 0. The above Taylor formula is of course still valid for any \( k \) and \( n \), but the terms are all zero after \( k+1 \). We again find a unique characterization of all functions on a certain set of nilpotents. This property will be stated in section {/placeholder1/} as a general axiom.

\section{Calculus of several variables}
We now take the usual course in calculus, which is to generalize the previous study to functions of more than one variable, i.e. \( R^n\to R \), and further on functions \( E\to V \) where \( E \) and \( V \) are ``vector spaces'' in a sense that will be made precise.

\subsubsection{Scalar functions}
Let \( f:R^n\to R \) be a function, and \( r=(r_1,\dots,r_n)\in R^n \). As one would expect, we calculate the partial derivatives by adding an infinitesimal increment along a single coordinate direction. That is, let \( g: D\to R \) be defined by
\begin{equation*}
  d\mapsto f(r_1+d,\dots,r_n)
\end{equation*}
By the KL axiom \ref{KL1}, there exists a unique \( b\in R \) such that
\begin{equation*}
  f(r_1+d,\dots,r_n) = f(r) + bd
\end{equation*}
We define \( \ddx{f}{x_1}(r_1,\dots,r_n)\) as \( b \), which simultaneously defines a function \( \ddx{f}{x_1}:R^n\to R \) Similarly, we define \( \ddx{f}{x_2},\dots,\ddx{f}{x_n} \). By iterating the process we obtain higher partial derivatives, denoted
\begin{equation*}
  \ddxk{k}{f}{x_{i_1}}{x_{i_k}}
\end{equation*}

A nice result is a parallel of Schwarz' theorem of the interchangeability of partial derivatives.

\begin{proposition}
  Let \( f:R^n\to R \) be a function. Then for any \( 1\leq i,j\leq n \)
  \begin{equation*}
    \ddxII{f}{x_i}{x_j}= \ddxII{f}{x_j}{x_i}
  \end{equation*}
  \label{prop:schwarz}
\end{proposition}
%TODO: this part looks horrible
The previous result follows directly from the following lemma which generalizes the second-order Taylor expansion to two dimensions.
\begin{lemma}
  Let \( f:R^n \to R \) be a function. Then for all \( (d_1,d_2)\in D\times D \)
  \begin{align*}
    f(r_1,\dots,r_i+d_1,\dots,r_j+d_2,\dots,r_n) = f(x_1,\dots,x_i,\dots,x_j+d_2,\dots,x_n) + \\
    \qquad + \ddx{f}{x_i}(x_1,\dots,x_i,\dots,x_j,\dots,x_n)d_1 + \ddxII{f}{x_i}{x_j}(x_1,\dots,x_n)d_2d_1 
  \end{align*}
\end{lemma}

\begin{proof}
  Apply the definition of the partial derivative in \( x_i \), and successively in \( x_j \)
  \begin{align*}
    f(r_1,\dots,r_i+d_1,\dots,r_j+d_2,\dots,r_n) & = f(x_1,\dots,x_i,\dots,x_j+d_2,\dots,x_n) + \\
                                                 & \qquad\qquad + \ddx{f}{x_i}(x_1,\dots,x_i,\dots,x_j+d_2,\dots,x_n)d_1 \\
                                                 & = f(x_1,\dots,x_n) + \ddx{f}{x_i}(x_1,\dots,x_n)d_1 + \ddxII{f}{x_i}{x_j}(x_1,\dots,x_n)d_2d_1
  \end{align*}
  and \( f(x_1,\dots,x_n), \ddx{f}{x_i}(x_1,\dots,x_n), \ddx{f}{x_j}(x_1,\dots,x_n), \ddxII{f}{x_i}{x_j}(x_1,\dots,x_n)  \) are the unique coefficients that satisfy this for universally quantified \( (d_1,d_2)\in D \).
\end{proof}

On the other hand, exchanging \( d_1 \) and \( d_2 \) (and commuting products and sums) in the above equations yields the same conclusion for \( \ddxII{f}{x_j}{x_i}(x_1,\dots,x_n) \). Since they are unique they must equal each other, proving (\ref{prop:schwarz}). By induction we also obtain that the order in which we differentiate higher partial derivatives does not matter.

\section{Vector functions}
We will now examine functions between \( R \)-modules. However, for a general \( R \)-module \( V \), it does not follow from the KL axiom (\ref{KL1}) that a version of it holds for functions \( D\to V \). In this case what we do is simply restrict our attention to those \( R \)-modules where this is the case. This leads to the following definition:

%TODO: example of R-module that isn't Euclidean.
\begin{defn}
  An \( R \)-module \( V \) is said to be a \emph{Euclidean \( R \)-module} if the vector form of the KL axiom holds. That is, for any function \( f: D\to V \),
  \begin{equation*}
    \exists! \, \vec a,\vec b\in V \text{ such that } f(d) = \vec a + d\vec b \,\forall d\in D
  \end{equation*}
  (again \( \vec a \) is evidently \( f(0) \))
\end{defn}

This may seem like a very ad-hoc definition, but we will encounter many Euclidean \( R \)-modules naturally. For example we have the following lemma.

\begin{lemma}
  \leavevmode
  \begin{enumerate}
    \item \( R^n \) is a Euclidean \( R \)-module.
    \item For any object \( X \), and Euclidean \( R \)-module \( V \), the exponential \( V^X \) is a Euclidean \( R \)-module. 
  \end{enumerate}
  \label{lm:Emod}
\end{lemma}

\begin{proof}
  \leavevmode
  \begin{enumerate}
    \item This is evident since a function \( f:D\to R^n \) is uniquely determined by each of its component functions. We let \( f_i = \pi_i \circ{f}  \) where \( \pi_i:R^n\to R \) is the \( i \)-\textit{th} coordinate projection. Then \( \vec a=(a_1,\dots,a_n), \vec b=(b_1,\dots,b_n) \) are the unique coefficients in the claim. Here each \( a_i,b_i \) is obtained by applying the KL axiom to \( f_i \).
    \item First of all, \( V^X \) is an \( R \)-module by the pointwise sum of maps. Let \( f:D\to V^X \) be a function. This defines, for each \( x\in X \), a function \( g(x):D\to V \) by
    \begin{equation*}
      d\mapsto f(d)(x)
    \end{equation*}
    since \( V \) is Euclidean, there exist unique \( a(x), b(x) \) such that
    \begin{equation*}
      f(d)(x) = \vec a(x) + d\vec b(x)\,\forall d\in D
    \end{equation*}
    Evidently, varying \( x \) defines functions \( \vec a,\vec b:X\to V \), which are unique such that \( f(d) = \vec a + d\vec b \,\forall d\in D\)
  \end{enumerate}
\end{proof}

Additionally, it's clear that any \( V \) which is isomorphic to \( R^n \) as an \( R \)-module is too Euclidean. In \cite{kock10} for example, it begins directly with this notion, calling them ``finite dimensional vector spaces''. Another fundamental example of a Euclidean \( R \)-module that we'll study are tangent spaces, in section {/placeholder2/}.

For now let us go back to doing calculus. The Euclidean structure allows one to define directional derivatives, as is done in \cite{lav96}, for example.
\begin{defn}
  Let \( V,E \) be Euclidean \( R \)-modules, \( f:V\to E \) a function and \( \vec u, \vec a\in V \). The derivative of \( f \) at \( \vec a \) in direction \( \vec u \) is the unique \( \vec b\in E \) such that
  \begin{equation*}
    \forall d\in D,\, f(\vec a + d\vec u) = f(\vec a) + d\vec b
  \end{equation*}
  We denote this by \( \partial_{\vec u}f(a) \). 
\end{defn}

We note that \( \partial_{(\farg)}f(\vec a) \) is a linear function. In other words, for all \( \lambda,\mu \in R\),
\begin{equation*}
  \partial_{\lambda \vec u + \mu \vec v}f(\vec a) = \lambda\partial_{\vec u}f(\vec a) + \mu\partial_{\vec v}f(\vec a)
\end{equation*}
The proof is a straightforward exercise (one may consult \cite[p. 13]{lav96}).

If \( V \) is finitely generated, i.e. \( V = \spn{\{e_1,\dots,e_n\}} \), then one can define
\begin{equation*}
  \ddx{f}{x_i}(\vec a) = \partial_{e_1}f(\vec a)
\end{equation*}
and if \( \vec u = \sum_{i=1}^{n}{u_ie_i} \) then by linearity
\begin{equation*}
  \partial_{\vec u}f(\vec a) = \sum_{i=1}^{n}\ddx{f}{x_i}u_i
\end{equation*}

Finally, the familiar total differential makes an appearance here too.

\begin{defn}
  Let \( f:V\to E \) be a function. The \emph{differential} of \( f \) at \( \vec a\in V \) is the function \( df(\vec a):V\to E \) defined by
  \begin{equation*}
    \vec u\mapsto \partial_{\vec u}f(\vec a)
  \end{equation*}
  \label{def:df(a)}
\end{defn}

As noted before, \( df(\vec a) \) defines a linear map \( V\to E \), for each \( a\in V \). It will also be useful to prove that, if \( f \) is linear, it is equal to its own differential. First we need a

\begin{lemma}
  Let \( V \) be an \( R \)-module, and \( E \) a Euclidean \( R \)-module. Let \( f:V\to E \) be such that
  \begin{equation*}
    f(\lambda \vec a) = \lambda f(\vec a)
  \end{equation*}
  for all \( \vec a\in V \) and \( \lambda \) in \( R \) (f is homogeneous). Then \( \forall \vec a\in V,\, df(\vec a) = df(\vec 0) \).
  \label{lm:homg}
\end{lemma}

\begin{proof}
  Let \( \vec a\in V \). Since \( E \) is Euclidean, we have
  \begin{equation*}
    f(\vec x + d\vec u) = f(\vec x) + d\partial_{\vec u}f(\vec x)
  \end{equation*}
  for all \( d\in D \) and \( \vec x,u\in V \). Letting \( x=\lambda a \) and \( d = \lambda\overline d \) for some \( \overline d\in D \), we find that
  \begin{equation*}
    f(\lambda\vec a + \lambda d\vec u) = f(\lambda\vec a)+\lambda d\partial_{\vec u}(\lambda\vec a)
  \end{equation*}
  At the same time, since \( f \) is homogeneous,
  \begin{align*}
    f(\lambda\vec a + \lambda d\vec u) &= \lambda f(\vec a + d\vec u) \\
                                       &= \lambda (f(\vec a) + d\partial_{\vec u}f(\vec a))
  \end{align*}
  Putting together the two equalities, we see that for all \( d\in D \),
  \begin{align*}
    \lambda d\partial_{\vec u}f(\lambda\vec a) = \lambda d\partial_{\vec u}f(\vec a) \Rightarrow \\
    \Rightarrow \lambda\partial_{u}f(\lambda\vec a) = \lambda\partial_{\vec u}(\vec a)
  \end{align*}
  Finally, we differentiate both sides with respect to \( \lambda \), which results in
  \begin{equation*}
    \partial_{\vec u}f(\lambda\vec a) + \lambda\frac{\partial}{\partial\lambda}(\partial_{\vec u}f(\lambda\vec a)) = \partial_{\vec u}f(\vec a)
  \end{equation*}
  Letting \( \lambda=0 \) concludes the proof.
\end{proof}

With that, we can quickly prove the following

\begin{proposition}
  Let \( V \) be an \( R \)-module, and \( E \) a Euclidean \( R \)-module. Let \( f:V\to E \) be homogeneous. Then \( f \) is linear and \( f = df(\vec 0) \).
  \label{prop:homg}
\end{proposition}

\begin{proof}
  Since \( E \) is Euclidean, for all \( \vec u\in V \) and \( d\in D \),
  \begin{align*}
    d\partial_{\vec u}f(\vec u) &= f(\vec u + d\vec u) - f(\vec u) \\
                                &= (1+d)f(\vec u) - f(\vec u)      \\
				&= d\cdot f(\vec u)
  \end{align*}
  by which
  \begin{equation*}
    f(\vec u) = \partial_{\vec u}f(\vec u) = \partial_{\vec u}(\vec 0) = df(\vec 0 )(\vec u)
  \end{equation*} 
\end{proof}

\chapter{The general KL axiom}

\section{A closer look at the first axiom} \label{ax1re}

Let \( (a,b)\in R\times R \). To this pair we can associate the function
\begin{equation*}
  f:D\to R
  \atop
  d\mapsto a + bd
\end{equation*}

In this manner we obtain a map
\begin{equation*}
  \alpha : R\times R\to R^D
\end{equation*}

Axiom \ref{KL1} can then be succintly stated by demanding that \( \alpha \) be bijective. Note that the \( R \)-module structure is also preserved, so that \( \alpha \) is an \( R \)-module isomorphism. Furthermore, \( R^D \) has a natural \( R \)-algebra structure given by the pointwise product of maps:

\begin{align}
  (a_1 + b_1d)(a_2 + b_2d) & = a_1a_2 + (a_1b_2 + b_1a_2)d + b_1b_2d^2 \\
                           & = a_1a_2 + (a_1b_2 + b_1a_2)d
  \label{R-alg}
\end{align}

As such, \( R\times R \) does not possess an \( R \)-algebra structure, but if we define a product by the above formula, that is
\begin{equation*}
  (a_1,b_1)\cdot(a_2,b_2) = (a_1a_2,a_1b_2 + b_1a_2)
\end{equation*}
then \( \alpha  \) is an \( R \)-algebra isomorphism.

It will be helpful for motivating the forthcoming definitions to state this in yet another way. The previous structure on \( R\times R \) is nothing more than the natural \( R \)-algebra on
\begin{equation*}
  W = \quot{R[X]}{(X^2)}
\end{equation*}
thus there is an isomorphism of \( R \)-algebras
\begin{equation*}
  W \stackrel{\alpha}{\cong} R^D
\end{equation*}

In the following section we will generalize this idea.

\section{Weil Algebras}

The infinitesimals \( D \) are one among many other ``small objects'' in SDG, called \emph{Weil algebras}, due to André Weil. We will use the general definition as per \cite{bun17}.

\begin{defn}
  A \emph{Weil algebra} \( W \) is a \( \Q \)-algebra together with a morphism
  \begin{equation*}
    \pi: W\to\Q
  \end{equation*}
  such that \( W \) is a local ring with maximal ideal \( I=\pi^{-1}(0) \) with \( I \) nilpotent and such that \( W \) is finite dimensional as a \( \Q \) vector space. A homomorphism of Weil algebras \( f:W_1\to W_2 \) is thus an algebra homomorphism sending the maximal ideal of \( W_1 \) to that of \( W_2 \).
\end{defn}



It will do to explore some simple consequences of this definition.

\begin{lemma}
  Any Weil algbra \( W \) may be finitely presented, that is, \( W \) is isomorphic to
  \begin{equation*}
    \Q[X_1,\dots,X_n]/(P_1,\dots,P_s)
  \end{equation*}
  where \( P_1,\dots P_s \) are polynomials of \( \Q[X_1,\dots,X_n] \).
  \label{lm:finpr}
\end{lemma}

\begin{proof}
  Let \( e_1,\dots,e_n \) be a \( \Q \)-basis of \( W \). The algbra structure is then determined by the constants \( \gamma_{ij}^k \) given by
  \begin{equation*}
    e_i\cdot e_j = \sum_{k=1}^n \gamma_{ij}^ke_k
  \end{equation*}
  in other words, the kernel of the surjection
  \begin{equation*}
    E: \Q[X_1,\dots,X_n]\to W
  \end{equation*}
  given by \( X_k\mapsto e_k \) is the ideal generated by the polynomials
  \begin{equation*}
    X_iX_j - \sum_{k=1}^n\gamma_{ij}^kX_k
  \end{equation*}
  With \( 1\leq i,j\leq n \). Thus \( W \) is isomorphic to \( \Q[X_1,\dots,X_n]/\ker E \) with \( \ker E \) finitely generated.
\end{proof}

Naturally, there is no reason for \( X_1,\dots,X_n \) to be a minimal set of generators. In \cite{lav96}, this minimal number of generators is referred to as the \emph{breadth} of a Weil algebra, and the smallest power that annihilates the maximal ideal as the \emph{height}.

%TODO:prove this
That a Weil algebra \( W \) is finite dimensional over \( \Q \) means that it is isomorphic to \( \Q^n \) with a certain product given by \( \gamma_{ij}^k \), for some \( n \). If \( A \) is any other \( \Q \)-algebra, then so can \( A^n \) be endowed with a product by letting \( e_ie_j = \gamma_{ij}^ke_k \) where \( e_l = (0,\stackrel{(l)}{\dots},1,\dots,0) \), making \( A^n \) an \( A \)-algebra. This structure is manifestly dependent on the \( e_l \). However, it does not depend on the presentation of \( W \). This algebra structure on \( A^n \) is denoted \( A[W] \). For instance, the \( R \)-algebra on \( R\times R \) from (\ref{R-alg}) arises in this way as \( R[W] \), where
\begin{equation*}
  W = \Q[X]/(X^2)
\end{equation*}
If we repeat the steps that we took to obtain a presentation of a Weil algebra \( W \cong \mathbb{Q}[X_1,\dots,X_n]/(P_1,\dots,P_s) \) then it's clear that for any other \( \mathbb{Q} \)-algebra \( A \), the algebra \( A[W] \) is also presented as \( A[X_1,\dots,X_n](P_1,\dots,P_s) \).

It's worth mentioning that in \cite{lav96}, the author defines Weil algebras directly as objects which according to our definition arise as \( R[W] \) with \( W \) a Weil algebra as defined above. The distinction is that \( R[W] \) need not be a local ring (for instance, (\ref{R-alg}) has the ideal \( ((0,a)) \) with \( a\in R \) which is not maximal since any element \( (d,0) \) with \( d\in D \) is in its complement, but it is not a unit. However this only changes the phrasing involved for the general KL axiom (the goal of this section), and not the content.

\section{Spectra of Weil Algebras}

The final piece involved in the statement of the general KL axiom is that of the \emph{spectrum} of a Weil algebra.
\begin{defn}
  Let \( W \) be a Weil algebra, and \( C \) be an \( R \)-algebra. If \( W \) is presented as 
  \begin{equation*}
    \Q[X_1,\dots,X_n]/(P_1\dots,P_s)
  \end{equation*}
  the \emph{spectrum} of \( W \) in \( C \) is
  \begin{equation*}
    \spec_C{W} = \{(a_1,\dots,a_n)\in C^n \mid P_j(a_1,\dots,a_n) = 0, j=1,\dots,s\} 
    \label{def:spec}
  \end{equation*}
\end{defn}
It is not difficult but it is prudent to check that this is well defined. In fact we shall prove that \( \spec_C(\farg) \) is functorial. Note that elements of \( \spec_C(W) \) can be ``evaluated'' at by classes in \( W \), since two representatives differ by an element of \( I=(P_1,\dots,P_j) \), but every polynomial in \( I \) is zero on \( \spec_C(W) \) by definition. We shorten such an evaluation, \( (P+I)(a) \), simply to \( P(a) \). 

Let \( W_1, W_2 \) be Weil algebras, with
\begin{align*}
  W_1 &\cong \Q[X_1,\dots,X_n]/I \\
  W_2 &\cong \Q[Y_1,\dots,Y_n]/J
\end{align*}

Now let \( f:W_1\to W_2 \) be a homomorphism of Weil algebras. Define
\begin{equation*}
  \spec_C(f):\spec_C(W_2)\to \spec_C(W_1)
\end{equation*}
by
\begin{equation*}
  \spec_C(f)(b_1,\dots,b_m) = (a_1,\dots,a_n)
\end{equation*}
where
\begin{equation*}
  a_i = f(X_i)(b_1,\dots,b_m)
\end{equation*}

We have
\begin{enumerate}
  \item \(\spec_C(\farg)\) preserves identities.
    In effect, the identity on a Weil algebra \( \id_W \) sends \( X_i \) to \( X_i \), and since \( X_i(b_1,\dots,b_m) = b_i \) so is \( \spec_C(\id_W) \) the identity on \( \spec_C(W) \).
  \item \(\spec_C(g\circ f) = \spec_C(f)\circ \spec_C(g)\), where defined. The proof is a simple verification.
\end{enumerate}

By virtue of the (contravariant) functoriality we get as a corollary that isomorphic Weil algebras induce a bijection on their spectra. With these definitions in place we may state the complete version of the KL axiom.

\section{The KL Axiom}

Let \( W\cong \mathbb{Q}[X_1,\dots,X_n]/I \) be a Weil algebra. Earlier we observed that we can evaluate (classes of) polynomials in \( W \) at elements of \( \spec_R(W) \) (just replace \( C \) with \( R \)). In other words each element of \( W \) defines a mapping \( \spec_R(W)\to R \)
\begin{equation*}
  W\to R^{\spec_R(W)}
\end{equation*}

In the same manner we also obtain a mapping
\begin{equation*}
  \alpha : R[W]\to R^{\spec_R(W)}
\end{equation*}

Again given by sending \( (p+I)\in R[W] \) to the map \( a\mapsto p(a) \). It is of course an algebra homomorphism. The KL Axiom asserts that it is bijective.
\begin{axiom}[KL]
  Let \( W\cong \mathbb{Q}[X_1,\dots,X_n]/I \) be a Weil algebra. Then the map
  \begin{equation*}
    \alpha : R[W]\to R^{\spec_R(W)}
  \end{equation*}
  given by \( p+I \mapsto (a\mapsto p(a)) \) is an \( R \)-algebra isomorphism.
  \label{ax:genKL}
\end{axiom}

By asserting this axiom we obtain many of the propositions in the first part of this monograph in a stronger version, and further generalizations. But before looking at those examples we'll note that the first KL axiom given is too a consequence of axiom \ref{ax:genKL}. This can be seen simply by applying the axiom to the Weil algebra \( \mathbb{Q}[X]/(X^2) \). The result is the equivalent version of axiom \ref{KL1} that we had already observed in section \ref{ax1re}.

What the general KL axiom provides is essentially, given a particular set of ``small'' elements of \( R \), a correspondence between functions from that set to \( R \) with a particular finite dimensional \( R \)-algebra. These ``small'' elelements are the objects \( \spec_R(W) \) for some Weil algebra \( W \).  We refer to them as ``small'' because they are meant to be a rigorization of the historically nebulous concept of ``infinitesimals''. We first introduced nilsquares (\( \spec_R(Q[X]/(X^2)) \)), on which all functions are affine, giving us first order derivatives. The Taylor formulas that came afterwards were restricted, however, to sums of elements of \( D \) and not any nilpotent element of \( R \). Since other nilpotents are too spectra of certain Weil algebra, we now have Taylor formulae for these, given by axiom \ref{ax:genKL}. We'll look at a few such spectra, study the KL axiom in regards to them and at the same time introduce their common nomenclature.

\begin{itemize}
  \item \( D_k := \spec_R(W_{k+1}) \) with \( W_{k+1} = \mathbb{Q}[X]/(X^{k+1}) \). These are elements of \( R \) whose \( (k+1) \)-th power is zero (in particular \( D_2 = D \)). One can check that \( W_{k+1} \) is isomorphic to the the set \( \{a_0+a_1\epsilon + \dots + a_k\epsilon^k\ \mid a_i\in R\} \) with \( \epsilon \) denoting the equivalence class of \( X \) in \( W_{k+1} \). By axiom \ref{ax:genKL} the set of functions \( D_k\to R \) are isomorphic as an \( R \)-algebra to the algebra \( W_{k+1} \). Another way of saying this is that:
    \begin{proposition}
      For all functions \( g:D_k\to R \) there exist unique coefficients \( a_0,\dots,a_k\in R \) such that
      \begin{equation*}
	 g(\delta) = a_0 + a_1\delta + \dots + a_k\delta^k \quad \forall\delta\in D_k
      \end{equation*}
    \end{proposition}
    And as an immediate consequence we have that, for any \( f:R\to R \) there exist unique coefficients \( a_0,\dots,a_k\in R \) such that
    \begin{equation*}
      f(x+\delta) = a_0 + a_1\delta + \dots + a_k\delta^k \quad \forall \delta\in R\, :\, \delta^{k+1}=0
    \end{equation*}
    That is, we have Taylor formulas of all orders (and one can check that the \( a_i \) are precisely the Taylor coefficients).

    %TODO: make sure of this
  \item \( (D_k)^n := \spec_R((W_{k+1})^n) \) where \( (W_{k+1})^n \) just means the cartesian product with itself \( n \) times (and so does taking the spectrum of the product of \( W_k \) coincide with the product of \( D_k \)). This is a Weil algebra, and it is presented by \( \mathbb{Q}[X_1,\dots,X_n](X_1^{k+1},\dots,X_n^{k+1}) \) . The general KL axiom applied to this particular algebra is what yields higher dimensional Taylor formulae. We state it now without proof (a tedious but simple combinatorial exercise).
    \begin{proposition}
      For any function \( f:R^n\to R \),
      \begin{equation*}
	f( \vec{x} + \vec{\delta} ) = \sum_{\alpha \leq k}{\frac{\vec{\delta}}{\alpha!}\cdot\frac{\partial^{\abs\alpha}f}{\partial x^\alpha}(\vec{x})}\quad \forall \delta : \delta^{k+1}=0
      \end{equation*}
    \end{proposition}
    where we're using the usual multi-index notation - if \( \alpha=(\alpha_1,\dots,\alpha_n) \) then \( \abs\alpha = \alpha_1+\dots+\alpha_n \), \( \alpha! = \alpha_1!\cdot\dots\cdot\alpha_n! \) and \( \frac{\partial^{\abs\alpha}f}{\partial x^\alpha}=\frac{\partial^{\abs\alpha}f}{\partial x^{\alpha_1}\dots\partial x^{\alpha_n}} \). The partial derivatives are defined in the same way as before.
  \item \( D(n) := \spec_R(W(n)) \) where \( W(n) = \Q[X_1,\dots,X_n]/(\{X_iX_j\}_{1\leq i,j \leq n}) \). A simpler way of writing this is \( D(n) = \{(d_1,\dots,d_n)\subset R^n\mid \text{the product of any two \( d_i,d_j \) is zero}\} \).
  \item \( D_k(n) := \spec_R(W_k(n)) \) where \( W_k(n) = \Q[X_1,\dots,X_n]/(\{X_{i_1}\cdots X_{i_k}\}_{1\leq i_1,\dots,i_k\leq n}) \). Again these are the elements of \( R^n \) such that the product of any \( k \) of their components is equal to zero.
\end{itemize}

There are some notable inclusions:

\begin{align*}
  D_k(n)  &\subset D_l(n)             \text{ iff k\(\leq\) l} \\
  D_k(n)  &\subset (D_k)^n            \\
  (D_k)^n &\subset D_{n\cdot k}(n)   
\end{align*}

\chapter{Microlinearity} \label{sec:microlinearity}

\section{A simple example}

We begin by observing a few specific cases. Let
\begin{equation*}
   D\times D \to D 
\end{equation*}
be the multiplication map, i.e.\ \( (d_1,d_2)\mapsto d_1d_2 \). Nothing in our theory suggests that this map should be surjective, but for many purposes it suffices to have the following property:

\begin{proposition}
  If \( f_1,f_2:D \to R \) verify that, for each \( d_1,d_2\in D,\, \allowbreak f_1(d_1d_2) = f_2(d_1d_2)  \) then \( f_1=f_2 \).
\end{proposition}

That is, it suffices to know any function \( f:D\to R \) on products of elements of \( D \). This is referred to in \cite{kock06} as ``\emph{\( R \) thinks that the multiplication : \( D\times D\to D \) is surjective}''.

\begin{proof}
  Define \( g = f_1-f_2 \). By our hypothesis \( g \) is identically zero on all elements \( d_1d_2 \) with \( d_1,d_2\in D \). By the KL axiom (since the previous section this refers to axiom \ref{ax:genKL}, but in this case axiom \ref{KL1} also suffices), there exist unique \( a,b\in R \) such that
  \begin{equation*}
    g(d)=a + bd\quad \forall d\in D 
  \end{equation*}
  On the one hand \( a=g(0)=g(0\cdot 0) = 0 \), so that \( g(d)=bd \). Thus, for all \( d_1,d_2\in D \) we have \( 0 = g(d_1d_2) = bd_1d_2 \), or in other words that \( bd_1d_2=0 \) for all \( d_1,d_2\in D \), which means that \( b=0 \).
\end{proof}

\section{A slightly more interesting example}

Consider two functions \( f,g:D\to X \) such that \( f(0)=g(0) \), where \( X \) is some set (object of \(  \sdgE) \). Here there is no reason to assume that this will define a map \( D(2)\to X \). It will be the case, however, if \( X \) is \( R \).

\begin{proposition}
  Let \( f,g:D\to R \) be functions such that \( f(0)=g(0) \). Then there exists a unique function \( h:D(2)\to R \) such that \( h(d,0) = f(d) \) and \( h(0,d)=g(d) \) for all \( d\in D \).
  \label{prop:d2}
\end{proposition}

\begin{proof}
  Let \( a_0 = g(0)=f(0) \). Then we have that for all \( d\in D \),
  \begin{align*}
    f(d) &= a_0 + a_1d \\
    g(d) &= a_0 + b_1d \\
  \end{align*}
  Thus, we simply define \( h \) by
  \begin{equation*}
    h(d_1,d_2) = a_0 + a_1d_1 + b_1d_2
  \end{equation*}
  This clearly satisfies the existence requirements. The Uniqueness is a consequence of the KL axiom. Let \( k \) be another function that satisfies the hypotheses. By the KL axiom applied to \( D(2) \),
  \begin{equation*}
    k(d_1,d_2) = \alpha_0 + \alpha_1d_1 + \alpha_2d_2
  \end{equation*}
  for unique \( \alpha_i \). By \( k(d,0)=f(d) \) we obtain \( \alpha_0=a_0 \) and \( \alpha_1 = a_1 \). Similarly \( \alpha_2=b_1 \). Since the \( \alpha_i \) are unique and determine \( k \), we get \( k=h \) (if preferred we could have concluded at the very definition of \( h \), since the KL axiom asserts that functions on \( D(2) \) are defined as such - the argument has simply been made explicit a bit further here).

\end{proof}

\section{Perceived colimits}

The previous examples can be stated in diagrammatic form, which will then lend itself to a generalization in terms of (co)limits. We'll focus on the second example. Consider the following commutative diagram:

\begin{equation}
  \xymatrix{
    {\{0\}} \ar[d]_0 \ar[r]^0   & D \ar[d]^{i_2} \\
    D \ar[r]_{i_1}              & D(2)
  }
  \label{dg:pushout}
\end{equation}
where \( i_1 \) and \( i_2 \) are the inclusion maps \( i_1(d)=(d,0),\, i_2(d)=(0,d) \) and \( 0 \) is the constant zero map. The previous observation that two maps \( f,g:D\to X \) do not have to define a map \( D(2)\to X \) is equivalent to saying that diagram \ref{dg:pushout} is not a pushout. But, if we apply the \( R^{(\farg)} \) functor to it, the result is:

\begin{equation*}
  \xymatrix{
    R^{\{0\}}       & R^D \ar[l]_{R^0}                           \\
    R^D \ar[u]^{R^0} & R^{D(2)} \ar[u]_{R^{i_2}} \ar[l]^{R^{i_2}} 
  }
  \label{dg:pullback}
\end{equation*}

\begin{proposition}
  Diagram \ref{dg:pullback} is a pullback.
  \label{prop:pullback}
\end{proposition}

%TODO:make sure of this and add to appendix
\begin{proof}
  First we will make the diagram a little neater. Firstly, \( R^{\{0\}} \) is isomorphic to \( R \). The map \( R^0 \) Takes \( f:D\to R \) and sends it to \( f\circ 0 \), which we simply identify with \( f(0) \) the same way we identified \( R^{\{0\}} \) with \( R \). Likewise, the map \( R^{i_1} \) sends \( g:D(2)\to R \) to \( g\circ i_1 \). Thus, for \ref{dg:pullback} to be a pullback, we'd need that the set \( R^D(2) \) be in bijection with the set of pairs \( f,g\in R^D \) such that \( f(0)=g(0) \). But this is precisely what we proved in proposition \ref{prop:d2}.
\end{proof}

How we refer to this phenomenon (or how it is often referred to) is \emph{``R perceives diagram \ref{dg:pushout} as a pushout}. This is of course generalizes, as we will see shortly. To do so we will give a few definitions first.

\begin{defn}
  Let \( \walg \) be the category of Weil algebras (a subcategory of the category of \( \Q \)-algebras \( \qalg \)). A \emph{good limit} of Weil algebras is a limit \( \left(L\stackrel{f_i}{\longrightarrow}D_i\right)_{i\in I} \) in \( \qalg \) which is also a limit in \( \walg \). That is, the morphisms in the diagram defining \( L \) must be Weil algebra homomorphisms, and so must the projections \( f_i \).
\end{defn}

\begin{defn}
  A diagram in \( \sdgE \) is said to be a (finite) \emph{quasi colimit} if it is the image under the functor
  \begin{equation*}
    \spec_R(\farg):\walg\to\sdgE
  \end{equation*}
  of a good finite colimit in \( \walg \).
\end{defn}

\begin{defn}
  An object \( X \) is said to perceive a quasi colimit as a colimit if the functor
  \begin{equation*}
    X^{(\farg)}:\sdgE\to\sdgE
  \end{equation*}
  takes said quasi colimit into a limit (of \( \sdgE \)).
\end{defn}

\begin{exmp}
  The following diagram is a good colimit of Weil algebras (it is a pushout in \( \qalg \), and the morphisms are Weil algebra homomorphisms).
  \begin{equation}
    \xymatrix{
      {\Q}                   & {\Q[X]/(X^2)} \ar[l]^0 \\
      {\Q[X]/(X^2)} \ar[u]^0 & {\Q[X,Y]/(X^2,Y^2,XY)} \ar[u]^{p_2} \ar[l]^{p_1}
    }
    \label{dg:goodlimex}
  \end{equation}
  where \( p_1(X)=X, p_1(Y)=0, p_2(X)=0 \), and \( p_2(Y)=X \). Via the \( \spec_R(\farg) \) functor this is taken to
  \begin{equation*}
    \xymatrix{
      {\{0\}} \ar[d]_0 \ar[r]^0 & D \ar[d]^{i_2} \\
      D \ar[r]_{i_1}            & D(2)
    } 
  \end{equation*}
  which is the same as diagram \ref{dg:pushout}. Thus, since diagram \ref{dg:goodlimex} is an example of a good finite limit, diagram \ref{dg:pushout} is a finite quasi colimit. Furthermore, \( R \) perceives \ref{dg:pushout} as a limit, as observed in proposition \ref{prop:pullback}.
\end{exmp}

%TODO: add other places where it is useful, if they arise
For plenty more examples, one should look at \cite{lav96}. The objective of these definitions are to state the upcoming proposition, which generalizes the previous observation that \emph{``R perceives\dots etc''}. The result will be of use in section \ref{sec:tangency} as a smoothness condition.

%TODO: converse
\begin{proposition}
  \( R \) perceives finite quasi colimits as colimits. Conversely, if \( R \) perceives a diagram as a colimit, it is a quasi colimit. 
  \label{prop:Rperc}
\end{proposition}

We'll need a simple result previously, in the form of the following

%TODO:proof
\begin{lemma}
  Let 
  \begin{equation*}
   \left(W_i\stackrel{p_i}{\longleftarrow}W_L\right)_{i\in I}
  \end{equation*}
  be any limit of Weil algebras. Then the diagram
  \begin{equation*}
    \left(R[W_i]\stackrel{p_i}{\longleftarrow}R[W_L]\right)_{i\in I}
  \end{equation*}
  is a limit of \( R \)-algebras.
  \label{lm:Rbracket}
\end{lemma}

With that, we can prove the proposition.

\begin{proof}[Proof of proposition \ref{prop:Rperc}]
  Let \( \left(L\stackrel{f_i}{\longleftarrow}D_i\right)_{i\in I} \) be a finite quasi colimit.\allowbreak That is, we may write the diagram as
  \begin{equation}
    \left(\spec_R(W_L)\stackrel{\spec_R(p_i)}{\longleftarrow}\spec_R(W_i)\right)_{i\in I}
    \label{dg:fquasicolt}
  \end{equation}
  where
  \begin{equation*}
   \left(W_i\stackrel{p_i}{\longleftarrow}W_L\right)_{i\in I}
  \end{equation*}
  is some good finite limit of Weil algebras. We need to prove that diagram \ref{dg:fquasicolt} is taken to a limit by the functor \( R^{(\farg)} \). Under this functor, the resulting diagram is
  \begin{equation}
    \left(R^{\spec_R(W_L)}\stackrel{R^{\spec_R(p_i)}}{\longrightarrow}R^{\spec_R(W_i)}\right)_{i\in I}
    \label{dg:Rfqc}
  \end{equation}
  Now we invoke the general KL axiom, which says that this diagram is isomorphic to
  \begin{equation*}
    \left(R[W_i]\stackrel{p_i}{\longleftarrow}R[W_L]\right)_{i\in I}
  \end{equation*}
  By lemma \ref{lm:Rbracket}, this is a limit, since by hypothesis
  \begin{equation*}
   \left(W_i\stackrel{p_i}{\longleftarrow}W_L\right)_{i\in I}
  \end{equation*} 
  is a limit of Weil algebras.
\end{proof}

\section{Microlinear objects}

The behavior of \( R \) studied above will be a fundamental requirement as we move onto studying other objects of synthetic differential geometry. We call these objects \emph{microlinear}, and it can be said that they are a sort of generalized ``manifold'' in the context of SDG. Let us state this precisely:

\begin{defn}
  An object \( M \) is said to be \emph{microlinear} if it perceives finite quasi colimits as limits.
\end{defn}

Thus, obviously \( R \) is microlinear. We'll make two observations.

\begin{proposition}
  If \( M \) is a microlinear object and \( X \) is any other object, then \( M^X \) is also microlinear.
  \label{prop:expmicro}
\end{proposition}

\begin{proposition}
  If
  \begin{equation*}
    \left( M\stackrel{f_\alpha}{\longrightarrow} M_\alpha \right)_{\alpha\in A}
  \end{equation*}
  is a limit, with \( M_\alpha \) microlinear for every \( \alpha\in A \), then \( M \) is microlinear.
  \label{prop:limmicro}
\end{proposition}


\chapter{Tangency}

At this point we have enough of the basic theory in place to begin doing actual \emph{geometry} on ``manifolds'' (microlinear objects).

\section{The tangent bundle}

Let \( M \) be a microlinear object and \( p\in M \). We make the following

\begin{defn}
  A tangent vector to \( M \) at \( p \) is a mapping 
  \begin{equation*}
    t:D\to M
  \end{equation*}
  such that \( t(0)=p \).
\end{defn}

We call the collection of all such tangent vectors \( T_pM \), the tangent space at \( p \) of \( M \). In resemblance to classical geometry, we of course expect each \( T_pM \) to have a vector space structure (and to justify calling them \emph{vectors} in the first place). It will not be so, since \( R \) is not a field in the classical sense, but we will prove that each \( T_pM \) is a Euclidean \( R \)-module. Let us begin by defining scalar multiplication:

\begin{equation*}
  R\times T_pM \to T_pM \atop (\lambda,t)\mapsto \lambda t
\end{equation*}

where the map \( \lambda t: D\to M \) is defined by

\begin{equation*}
  (\lambda t)(d) = t(\lambda d)
\end{equation*}

As for addition, consider two tangent vectors at \( p \), \( t_1,t_2:D\to M \). We now use that \( M \) is microlinear. Recall from earlier that the diagram below is a quasi pushout:

\begin{equation}
  \xymatrix{
    {\{0\}} \ar[d]_0 \ar[r]^0   & D \ar[d]^{i_2} \\
    D \ar[r]_{i_1}              & D(2)
  }
  \label{dg:pushoutcopy}
\end{equation}

Since \( M \) is microlinear, it perceives \ref{dg:pushoutcopy} as a pushout. As we studied in the case of \( R \), this means that the maps \( D(2)\to M \) are in bijection with pairs of maps \( D\to M \) that are equal at \( 0 \). This is the case with \( t_1 \) and \( t_2 \), since \( t_1(0)=p=t_2(0) \). Thus, there exists a unique map, which we call

\begin{equation*}
  s_{t_1,t_2}:D(2)\to M
\end{equation*}

allowing us to define

\begin{equation*}
  t_1+t_2:d\mapsto s_{t_1,t_2}(d,d)
\end{equation*}

Finally, let us define \( 0\in T_pM \) as the constant map \( 0(d)=p \,\forall d\in D \), and the additive opposite of a vector \( t \) as \( -t \), given by \( (-t)(d) = t(-d) \).

\begin{proposition}
  Let \( p\in M \). With the operations defined above, \( T_pM \) is an \( R \)-module.
\end{proposition}

\begin{proof}
  \begin{enumerate}
    \item Addition is commutative. Let \( t_1,t_2\in T_pM \). Again, this defines
      \begin{equation*}
	s_{t_1,t_2}:D(2)\to M
      \end{equation*}
      as the unique map satisfying \( s_{t_1,t_2}(d_1,0) = t_1(d_1) \) and \( s_{t_1,t_2}(0,d_2) = t_2(d_2) \). On the other hand, \( t_1 \) and \( t_2 \) also define the map
      \begin{equation*}
	s_{t_2,t_1}:D(2)\to M
      \end{equation*}
      unique among maps satisfying \( s_{t_2,t_1}(d_2,0) = t_2(d_2) \) and \( s_{t_2,t_1}(0,d_1) = t_1(d_1) \). Therefore, we have that for all \( (d_1,d_2)\in D(2) \)
      \begin{equation*}
	s_{t_1,t_2}(d_1,d_2) = s_{t_2,t_1}(d_2,d_1)
      \end{equation*}
      In particular, \( s_{t_1,t_2} \) and \( s_{t_2,t_1} \) are equal on the diagonal \( \left\{ (d,d)\mid d\in D \right\}\subset D(2) \). So, by definition of \( t_1+t_2 \) and \( t_2+t_1 \), these two are equal.

    \item Addition is associative. This requires a generalization of proposition \ref{prop:pullback}, that is, that \( R \) perceives the diagram
      \begin{equation}
	\xymatrix{
	  {\{0\}} \ar[d]_0 \ar[r]^0   & D(q) \ar[d]^{i_2} \\
	  D(p) \ar[r]_{i_1}              & D(p+q)
	}
	\label{dg:pushoutgen}
      \end{equation}
      as a pushout, for any \( p,q\geq 1 \). Here the maps \( i_1,i_2 \) are
      \begin{align*}
	i_1(d_1,\dots,d_p) &= (d_1,\dots,d_p,0,\dots,0) \\
	i_2(d_1,\dots,d_q) &= (0,\dots,0,d_1,\dots,d_q)
      \end{align*}
      The proof is nearly identical to the case we have examined (\( p=q=1 \)), one can refer to \cite[p. 48]{lav96} for details. That \( R \) perceives \ref{dg:pushoutgen} as a pushout means that \( M \) does as well, \( M \) being microlinear. In turn, this implies that the diagram
      \begin{equation*}
	\xymatrix{
	  M                & M^{D(2)} \ar[l]_{M^0} \\
	  M^D \ar[u]^{M^0} & M^{D(3)} \ar[u]_{M^{i_2}} \ar[l]^{M^{i_1}}
	}
      \end{equation*}
      is a pullback. That is, any function
      \begin{equation*}
	g:D(3)\to M \atop (d_1,d_2,d_3)\mapsto g(d_1,d_2,d_3)
      \end{equation*}
      is uniquely determined by the functions
      \begin{equation*}
	g_1:D\to M \atop d_1\mapsto g(d_1,0,0)
      \end{equation*}
      and
      \begin{equation*}
	g_{23}:D(2)\to M \atop (d_2,d_3) \mapsto g(0,d_2,d_3)
      \end{equation*}
      Now, by applying the characterization of maps \( D(2)\to M \) on \( g_{23} \), we obtain that \( g \) is the unique map satisfying \( g(d_1,0,0)=g_1(d_1) \), \( g(0,d_2,0)=g_{23}(d_2,0) \), and \( g(0,0,d)=g_{23}(0,d_3) \) for all \( (d_1,d_2,d_3)\in D(3) \). In other words, any map
      \begin{equation*}
	g:D(3)\to M
      \end{equation*}
      is uniquely determined by three maps \( g_1,g_2,g_3:D\to M \) with \( g_1(0)=g_2(0)=g_3(0) \), and such that
      \begin{align*}
	g_1(d_1) &= g(d_1,0,0) \\
	g_2(d_2) &= g(0,d_2,0) \\
	g_3(d_3) &= g(0,0,d_3)
      \end{align*}
      With that said, let \( t_1,t_2,t_3\in T_pM \) be three tangent vectors. As before, \( t_1 \) and \( t_2 \) determine a unique map
      \begin{equation*}
	s_{t_1,t_2}:D(2)\to M
      \end{equation*}
      So that, together with \( t_3 \), they determine a map \( g_l:D(3)\to M \) given by
      \begin{align*}
	g_l(d_1,0,0) &= s_{t_1,t_2}(d_1,0) \\
	g_l(0,d_2,0) &= s_{t_1,t_2}(0,d_2) \\
	g_l(0,0,d_3) &= t_3(d_3)
      \end{align*}
      Likewise, we obtain a map \( g_r \) by
      \begin{align*}
	g_r(d_1,0,0) &= t_1(d_1)           \\
	g_r(0,d_2,0) &= s_{t_2,t_3}(d_2,0) \\
	g_r(0,0,d_3) &= s_{t_2,t_3}(0,d_3)
      \end{align*}
      But these are the same map, since they both satisfy
      \begin{align*}
	g_l(d_1,0,0) &= g_r(d_1,0,0) = t_1(d_1) \\
	g_l(0,d_2,0) &= g_r(0,d_2,0) = t_2(d_2) \\
	g_l(0,0,d_3) &= g_r(0,0,d_3) = t_3(d_3)
      \end{align*}
      Now note that
      \begin{align*}
	((t_1+t_2)+t_3)(d) &= s_{t_1+t_2,t_3}(d,d) = g_l(d,d,d) \\
	(t_1+(t_2+t_3))(d) &= s_{t_1,t_2+t_3}(d,d) = g_r(d,d,d) 
      \end{align*}
      by which we conclude that \( ((t_1+t_2)+t_3) = (t_1+(t_2+t_3)) \).

    \item The map 0 is the additive identity. Let \( t:D\to M \) be a tangent vector at \( p \), and \( 0:D\to M \) be the constant \( p \). The identities
      \begin{align*}
	s_{t,0}(d,0) &= t(d) \\
	s_{t,0}(0,d) &= 0(d) = p
      \end{align*}
      uniquely determine \( s_{t,0}:D(2)\to M \). But the function \( t\circ \pi_1:D(2)\to M \), where \( \pi_2:D(2)\to D \) is the projection onto the second coordinate, also satisfies those equations, making \( s_{t,0} = t\circ \pi_2 \). Therefore,
      \begin{equation*}
	(t+0)(d) = s_{t,0}(d,d) = (t\circ \pi_2)(d,d) = t(d)
      \end{equation*}

    \item The opposite of a vector, as defined, satisfies \( t + (-t) = 0 \). The function \( s_{t,-t}:D(2)\to M \) is unique such that \( s_{t,-t}(d,0) = t(d) \) and \( s_{t,-t}(0,d) = (-t)(d) = t(-d) \). Again, we explicitly exhibit the function
      \begin{equation*}
	f:D(2)\to M \atop (d_1,d_2)\mapsto t(d_1-d_2)
      \end{equation*}
      and \( f \) satisfies the same equations, making it equal to \( s_{t,-t} \). Therefore, for any \( d\in D \)
      \begin{equation*}
	(t+(-t))(d) = s_{t,-t}(d,d) = f(d,d) = t(d-d) = t(0) = p
      \end{equation*}

    \item We have the identities
      \begin{enumerate}
	\item\label{a} \( (\alpha + \beta)t = \alpha t + \beta t \)
	\item\label{b} \( \alpha(t_1+t_2) = \alpha t_1 + \alpha t_2\)
	\item\label{c} \( \alpha (\beta t) = (\alpha \beta) t\)
	\item\label{d} \( 1\cdot t = t \)
      \end{enumerate}
      With \( \alpha,\beta\in R \) and \( t,t_1,t_2\in T_pM \). The tangent vector \( \alpha t + \beta t \) is the unique map \( s_{\alpha t, \beta t}:D(2)\to M \) satisfying
      \begin{align*}
	s_{\alpha t, \beta t}(d_1,0) &= (\alpha t)(d_1)= t(\alpha d_1) \\
	s_{\alpha t, \beta t}(0,d_2) &= (\beta t)(d_2)= t(\beta d_2)
      \end{align*}
      On the other hand, the map \( c \), defined by
      \begin{equation*}
       c(d_1,d_2) = t(\alpha d_1 + \beta d_2)
      \end{equation*}
      satisfies the equations, by which \( s_{\alpha t, \beta t} = c\). So,
      \begin{equation*}
	(\alpha t + \beta t)(d) = c(d,d) = t(\alpha d + \beta d) = t((\alpha + \beta)d) = (\alpha + \beta)t(d)
      \end{equation*}
      by definition. That proves \ref{a}. To prove \ref{b} we make an analogous argument, this time considering
      \begin{equation*}
	c(d_1,d_2) = s_{t_1,t_2}(\alpha d_1, \alpha d_2)
      \end{equation*}
      leaving \ref{c} and \ref{d}, which are immediate.
  \end{enumerate}
  This concludes the proof that \( T_pM \) is an \( R \)-module. We will now prove that it is Euclidean. Recall that this means that, for any map \( \varphi:D\to T_pM \), there should be a unique vector \( t\in T_pM \) such that, for all \( d\in D \)
  \begin{equation*}
    \varphi(d) = \varphi(0) + d\cdot t
  \end{equation*}
  This will again be a consequence of microlinearity. Define
  \begin{equation*}
    \tau : D\times D\to M \atop (d_1,d_2)\mapsto (\varphi(d_1)-\varphi(0))(d_2)
  \end{equation*}
  Note that \( \varphi(d_1)-\varphi(0) \) is a tangent vector at \( p \), so that
  \begin{equation*}
    \tau(d_1,0) = \tau(0,d_2) = \tau(0,0) = p
  \end{equation*}
  for all \( d_1,d_2\in D \). We'll now need another lemma concerning a quasi colimit.

  \begin{lemma}
    M perceives the diagram
    \begin{equation}
      \xymatrix{
	D \ar@<1em>[r]^{i_1} \ar[r]^{i_2} \ar@<-1em>[r]^{0} & D\times D \ar[r]^{\mu} & D
      }
      \label{dg:3coequ}
    \end{equation}
    \label{lm:3coequ}
    as a colimit (a ``triple coequalizer'', if one wishes). Here the maps \( i_1,i_2 \) are defined by \( i_1(d)=(d,0), i_2(d)=(0,d) \), \( 0 \) is the zero map, and \( \mu \) is the multiplication of elements of \( R \), restricted to elements of \( D \).
  \end{lemma}

  In other words, the diagram
  \begin{equation*}
    \xymatrix{
      M^D & M^{D\times D} \ar@<1ex>[l]_{M^{i_1}} \ar[l]_{M^{i_2}} \ar@<-1ex>[l]_{M^{0}} & M^D \ar[l]_{M^{\mu}} 
    }
  \end{equation*}
  is a limit (a triple equalizer). That is, if \( g\in M^{D\times D} \) is such that
  \begin{equation}
    g(d_1,0)=g(0,d_2)=g(0,0) \quad \forall d_1,d_2\in D
    \label{eq:3equ}
  \end{equation}
  then there exists a unique mapping 
  \begin{equation*}
    t:D\to M
  \end{equation*}
  such that
  \begin{equation*}
    g(d_1,d_2) = (t\circ \mu)(d_1,d_2) = t(d_1d_2)
  \end{equation*}
  
  This is all we need, since the map \( \tau \) verifies the equations \ref{eq:3equ}. Therefore there exists a unique \( t:D\to M \) with
  \begin{equation*}
    \tau(d_1,d_2)=t(d_1d_2) \quad \forall d_1,d_2\in D
  \end{equation*}
  In other words for all \( d_1,d_2 \),
  \begin{equation*}
    (\varphi(d_1) - \varphi(0))(d_2) = t(d_1d_2) = (d_1t)(d_2)
  \end{equation*}
  where the last equality is by definition of scalar multiplication of tangent vectors. The above holds in particular for all \( d_2\in D \), so that there is an equality of maps
  \begin{equation*}
    \varphi(d_1) = \varphi(0)+d_1t
  \end{equation*}
\end{proof}.

The proof of lemma \ref{lm:3coequ} is fairly straightforward. Since \( M \) is microlinear, it amounts to proving that diagram \ref{dg:3coequ} is a quasi colimit.

\begin{proof}[Proof of lemma \ref{lm:3coequ}]
  The diagram of Weil algebras
  \begin{equation}
    \xymatrix{
      \Q[X]/(X^2) & \Q[X,Y]/(X^2,Y^2) \ar@<3ex>[l]_{f_1} \ar[l]_{f_2} \ar@<-3ex>[l]_{f_3} & \Q[X]/(X^2) \ar[l]_{m} 
    }
    \label{dg:3equWalg}
  \end{equation}
  is a limit. Here the functions \( m,f_1,f_2,f_3 \) are defined by
  \begin{align*}
    f_1: & \begin{array}{c@{\hspace{0.3em}}l} X & \mapsto X \\ Y & \mapsto 0 \end{array} \\[4ex]
    f_2: & \begin{array}{c@{\hspace{0.3em}}l} X & \mapsto 0 \\ Y & \mapsto X \end{array} \\[4ex]
    f_3: & \begin{array}{c@{\hspace{0.3em}}l} X & \mapsto 0 \\ Y & \mapsto 0 \end{array} \\[4ex]
    m:   & \begin{array}{c@{\hspace{0.3em}}l} X & \mapsto XY                 \end{array} 
  \end{align*}
  To begin with, the diagram obviously commutes. Now, let \( A \) be any object, and \( g:A\to \Q[X,Y]/(X^2,Y^2) \) a map such that \( f_i\circ g = f_j\circ g \) for any \( i,j \) (\( g \) makes a similar diagram commute). Let \( a\in A \). Its image under \( g \) is an element of \( \Q[X,Y]/(X^2,Y^2) \), and such can be written, modulo \( (X^2,Y^2) \) as
  \begin{equation*}
    g(a) = c_{00} + c_{10}X + c_{01}Y + c_{11}XY
  \end{equation*}
  The condition that \( g \) commutes with the \( f_i \) force \( c_{10}=c_{01}=0 \). Define
  \begin{equation*}
    h(a) = c_{00} + c_{11}X
  \end{equation*}
  By varying \( a \) this defines a map \( h:A\to \Q[X]/(X^2) \). Since \( c_{00},c_{11} \) are unique modulo \( (X^2,Y^2) \), the map \( h \) is the unique map such that the following diagram commutes:
  \begin{equation*}
    \xymatrix{
      \Q[X]/(X^2) & \Q[X,Y]/(X^2,Y^2) \ar@<3ex>[l]_{f_1} \ar[l]_{f_2} \ar@<-3ex>[l]_{f_3} & \Q[X]/(X^2) \ar[l]_{m} \\
      &                                                                                   & A \ar[lu]^g \ar@{.>}[u]^h
    }
  \end{equation*}
  This concludes that diagram \ref{dg:3equWalg} is a limit. It is left to the reader to check that diagram \ref{dg:3coequ} is the result of applying the \( \spec_R \) functor to \ref{dg:3equWalg}, making it a quasi colimit. Since \( M \) was microlinear, the lemma is proven.  
\end{proof}

The natural next step is to define the differential (or derivative, or tangent map, etc.) of a mapping between two microlinear objects. Let \( M,N \) be microlinear, and \( f:M\to N \) a map. Let \( p\in M \). We define:
\begin{equation*}
  df_p : T_pM\to T_{f(p)}N
\end{equation*}
by
\begin{equation*}
  df_p(t) = f\circ t
\end{equation*}

As we should expect, this map is linear. Since \( T_pM \) is Euclidean, by proposition \ref{prop:homg} it suffices to see that it is homogeneous. For \( d\in D \), we have
\begin{equation*} 
  df_p(\alpha t)(d) = (f\circ (\alpha t))(d) = f(t(\alpha d)) = (f\circ t)(\alpha d) = (\alpha \cdot df_p(t))(d)
\end{equation*}

Another desirable property is that if \( V \) is a Euclidean \( R \)-module, it is canonically isomorphic to \( T_pV \) at every \( p\in V \). This is easy to see by presenting the isomorphism explicitly. Define
\begin{equation*}
  \lambda_p:V\to T_pV
\end{equation*}
by sending \( v\in V \) to the map
\begin{equation*}
  d\mapsto p+dv
\end{equation*}
This is a bijection since, by virtue of \( V \) being Euclidean, every map \( t:D\to V \) is characterized by a unique \( b\in V \) such that
\begin{equation*}
  d\mapsto t(0)+db
\end{equation*}
In \( T_pV \), each \( t(0) \) is equal to \( p \), so the inverse of \( \lambda_p \) is \( \lambda_p^{-1}(t) = b \). The map \( \lambda_p \) is obviously homogeneous, so again by proposition \ref{prop:homg} it is linear. As a short exercise, one can prove that, under this identification, \( df_p \) corresponds to \( df(p) \) (definition \ref{def:df(a)}) for maps of Euclidean \( R \)-modules \( f:V\to E \).

\section{Vector bundles; the tangent bundle}

\begin{defn}
  Let \( \pi:E\to M \) be a mapping of microlinear objects. We say that \( \pi \) is a \emph{(resp. Euclidean) vector bundle} if each fiber \( \pi_p = \pi^{-1}(\{p\}) \) of \( \pi \) is a (resp. Euclidean) \( R \)-module.
\end{defn}

As a prominent example, we have

\begin{defn}
  Let \( M \) be a microlinear object. The \emph{tangent bundle} on \( M \) is given by
  \begin{equation*}
    \pi:M^D\to M \atop t\mapsto t(0)
  \end{equation*}
\end{defn}

The previous map indeed defines a Euclidean vector bundle. First, \( M^D \) is microlinear by proposition \ref{prop:expmicro}, and second each fiber is just \( T_pM \), which we have just seen to be a Euclidean \( R \)-module.

We have defined vector bundles, so we should define what it means to be a morphism of vector bundles. If
\begin{align*}
  \pi_1:E_1\to M_1 \\
  \pi_2:E_2\to M_2 \\
\end{align*}
are two vector bundles, then a pair \( (\varphi,f) \) is said to be a morphism of the vector bundles \( \pi_1,\pi_2 \) if the following diagram commutes:
\begin{equation*}
  \xymatrix{
    E_1 \ar[d]_{\pi_1} \ar[r]^\varphi & E_2 \ar[d]^{\pi_2} \\
    M_1                \ar[r]_f       & M_2
  }
\end{equation*}
Equivalently, \( \varphi \) takes the fiber at \( p\in M_1 \) to the fiber at \( f(p)\in M_2 \).

Again, an important example is provided by the tangent bundle to a microlinear object. If \( M,N \) are microlinear objects, and \( f:M\to N \) is a map, then \( (f^D,f) \) is a morphism of the tangent bundle at \( M \) to the tangent bundle at \( N \), as evidenced by the commutative diagram
\begin{equation*}
  \xymatrix{
    M^D \ar[d]_{\pi_M} \ar[r]^{f^D} & N^D \ar[d]^{\pi_N} \\
    M                  \ar[r]_f     & N
  }
\end{equation*}
Where \( \pi_M,\pi_N \) are the projections defining the tangent bundles of \( M,N \), respectively. If we define \( \mlin \) to be the category of microlinear objects (with morphisms regular maps), and \( \vbun \) the category of vector bundles (with morphisms of vector bundles as previously defined), then the association of each microlinear object to its tangent bundle defines a functor
\begin{equation*}
  T:\mlin\to\vbun
\end{equation*}
with
\begin{align*}
  TM &= M^D     \\
  Tf &= (f^D,f) \\
\end{align*}

which we call the \emph{tangent functor}. Note that fiberwise, this is exactly the association of each object to its tangent space at a point \( p \), and each map \( f \) to its differential \( df_p \).

\section{Vector Fields}

As in classical differential geometry, we define a \emph{section} of a vector bundle to be a right inverse to the projection map of the bundle. That is, if \( \pi:E\to M \) is a vector bundle, then a section of \( \pi \) is a map \( s:M\to E \) such that \( \pi\circ s = \id_M \). We will write \( \Gamma(\pi) \) for the set of sections of \( \pi \).

Let \( \pi:E\to M \) be a vector bundle. It's immediate that \( \Gamma(\pi) \) comes equipped with an \( R \)-module structure, by the \( R \)-module structure on each fiber: define \( (s_1+s_2)(p) = s_1(p)+s_2(p) \), and so on (where \( s_1,s_2 \) are sections). Furthermore we have the following proposition.

\begin{proposition}
  The set \( \Gamma(\pi) \) is microlinear, and if \( \pi \) is a Euclidean vector bundle, \( \Gamma(\pi) \) is a Euclidean \( R \)-module.
\end{proposition}

\begin{proof}
  The objects \( E \) and \( M \) are microlinear by definition of a vector bundle, and by proposition \ref{prop:expmicro} so are \( E^M, M^M \) microlinear. Now observe that the diagram
  \begin{equation*}
    \xymatrix{
      {\Gamma(\pi)\,\,} \ar@{>->}[r] & E^M \ar@<1ex>[r]^{\varphi} \ar@<-1ex>[r]_{I} & M^M 
    }
  \end{equation*}
  is an equalizer, where the first arrow is just the inclusion of \( \Gamma(\pi) \) into \( E^M \), \( \varphi(s) = \pi\circ s \), and \( I(s) = \id_M \). Thus \( \Gamma(\pi) \) is a limit of microlinear objects, meaning it is microlinear by proposition \ref{prop:limmicro}. Now suppose that \( \pi \) is a Euclidean vector bundle, and let \( f:D\to \Gamma(\pi) \) be a map. For fixed \( p\in M \), the Euclidean \( R \)-module structure on \( \pi_p \) gives that there exists a unique \( b(p)\in \pi_p \) (we make explicit the dependence on \( p \)) such that for all \( d\in D \)
  \begin{equation*}
    f(d)(p) = f(0)(p) + d\cdot b(p)
  \end{equation*}
  Varying \( p \) thus gives a section \( b:M\to E \) (it is a section, since \( b(p)\in \pi_p = \pi^{-1}(p) \)) which is unique such that
  \begin{equation*}
    f(d) = f(0) + d\cdot b
  \end{equation*}
  and so \( \Gamma(\pi) \) is Euclidean.
\end{proof}

We should also make the observation that \( \Gamma(\pi) \) is a module over \( R^M \), as well. For \( f:M\to R \) and \( s\in \Gamma(\pi) \) define

\begin{equation*}
  (f\cdot s)(p)=f(p)\cdot s(p)
\end{equation*}

We can now define what a vector field is, in a much expected way.

\begin{defn}
  Let \( M \) be a microlinear object. A \emph{vector field} on \( M \) is a section of the tangent bundle on \( M \). We denote the set of vector fields by \( \vfld M \).
\end{defn}

Now since the tangent bundle is an exponential object, we can can automatically establish a correspondence between three familiar conceptions of a vector field. On the first hand a vector field is a map
\begin{equation*}
  X:M\to M^D
\end{equation*}
such that \( X(p)(0)=p \) for every \( p\in M \). But then \( X \) is equivalent to giving a map
\begin{equation*}
  X:M\times D \to M
\end{equation*}

such that \( X(p,0)=p \). Thus, the notion of infinitesimal flow is recovered. Furthermore we have the principal of superposition.

\begin{proposition}
Let \( X:M\times D\to M \) be a vector field. Then for every \( p\in M \) and \( (d_1,d_2)\in D(2) \)
\begin{equation*}
  X(p,d_1+d_2) = X(X(p,d_1),d_2)
\end{equation*}
\label{prop:superpos}
\end{proposition}

\begin{proof}
  By fixing \( p \) we obtain two maps \( f,g:D(2)\to M \), which are defined by
  \begin{align*}
    f(d_1,d_2) &= X(p,d_1+d_2)    \\
    g(d_1,d_2) &= X(X(p,d_1),d_2) \\
  \end{align*}
  Since \( X \) is a vector field we have the equalities
  \begin{align*}
    f(d,0) &= g(d,0) \\
    f(0,d) &= g(0,d) \\
  \end{align*}
  hence, since \( M \) is microlinear, \( f \) and \( g \) are the same map.
\end{proof}

Lastly, we also have the notion of a vector field as an ``infinitesimal transformation of the identity''. That is, from a map
\begin{equation*}
  X:D\times M\to M
\end{equation*}
there corresponds a map (which we'll call by the same name)
\begin{equation}
  X:D\to M^M
  \label{eq:vecfield3}
\end{equation}
%TODO: make sure not all flows are diffeo
and it is such that \( X(0)=\id_M \).

From here on we will use the three ways of giving a vector field interchangeably. As in \cite{lav96} we refer to the image of an element \( d\in D \) by a map such as \ref{eq:vecfield3}, as \( X_d:M\to M \). For \( p\in M \), we should visualize \( X_d(p) \) as the result of letting \( p \) flow along the vector field for \( d \) units of ``time''. Whereas in classical differential geometry the flow corresponding to a vector field need not define a group of transformations (but it does a semi-group), it will be the case here.

\begin{proposition}
  Let \( X:M\to M \) be a vector field. Then for all \( d\in D \), \( X_d \) is a bijection and \( X_d^{-1}=X_{-d} \).
  \label{prop:inftrans}
\end{proposition}

\begin{proof}
  This is a simple consequence of proposition \ref{prop:superpos}, since
  \begin{equation*}
    X(X(p,d),-d)=X(p,0)=p
  \end{equation*}
  for all \( p\in M \).
\end{proof}

The following proposition also establishes a strong parallel with classical Lie algebras of vector fields (although the ``Lie'' part will be seen later).

\begin{proposition}
  Let \( M \) be microlinear. Denote the set of invertible maps from \( M \) to itself as \( \aut(M) \). This is a microlinear object, and \( \vfld M \) is \( R \)-module isomorphic to \( T_{\id_M}\aut(M) \), the tangent space at the identity.
\end{proposition}

\begin{proof}
  To prove that \( \aut(M) \) is microlinear, we proceed as usual to prove that it is a limit of microlinear objects. To do this, note that \( \aut(M) \) is equal to the set
  \begin{equation*}
    \{ (f,g) \in M^M\times M^M \mid g\circ f = f \circ g = \id_M \}
  \end{equation*}

  In other words, we have a diagram
  \begin{equation*}
    \xymatrix{
      {\aut(M)} \ar@{>->}[r]^{i} & M^M\times M^M \ar@<1ex>[r]^{c_1} \ar[r]^{c_2}\ar@<-1ex>[r]^{I} & M^M
    }
  \end{equation*}
  where \( i \) is the mapping \( f\mapsto (f,f^{-1}) \), \( c_1 \) is the mapping \( (f,g)\mapsto f\circ g \), \( c_2 \) is \( (f,g)\mapsto g\circ f \), and \( I \) is \( (f,g)\mapsto \id_M \). The diagram commutes, and any object mapping into \( M^M\times M^M \) in such a way will clearly be in bijection with (a subset of) \( \aut(M) \). And, of course, \( M^M\times M^M \) along with \( M^M \) are microlinear, as per proposition \ref{prop:expmicro}. Now, any element \( t \) of \( T_{\id_M}\aut(M) \) is clearly associated with an element of \( T_{\id_M}(M^M) \), namely the ``same'' map, extending the codomain. Thus, it is an element of \( \vfld M \), thanks to the last interpretation of vector fields we observed above. Conversely, by proposition \ref{prop:inftrans}, any element of \( \vfld M \) maps \( D \) to invertible maps in \( M^M \), in other words elements of \( \aut(M) \). In summary, the map \( \vfld M \to T_{\id_M}\aut(M) \) given by
  \begin{equation*}
    (t:D\to M^M) \mapsto (\bar t:D\to \aut(M))
  \end{equation*}
  is a bijection. It is trivially homogeneous (\( \lambda t \mapsto \lambda \bar t \)), which means it is linear, thanks to proposition \ref{prop:homg}.
\end{proof}

\section{The Lie algebra of vector fields}

In classical differential geometry, the Lie bracket of two vector fields is often thought of a commutator of the ``flows'' induced by the vector fields (and in the case of matrix Lie groups the Lie bracket is indeed represented by matrix commutators). The ``problem'', as noted earlier, is that classical flows may not correspond to permutations of the manifold (again, in general only a semi-group of transformations). That isn't to say that differential geometers suffer greatly from this, but we again take the opportunity to show how synthetic differential geometry actually behaves how we would like to \emph{think} about geometry behaving. That is, the Lie bracket will be directly associated to the commutator of the bijections defined by their flows.

Let \( X,Y \) be vector fields on \( M \). They each define bijections \( X_d,Y_d:M\to M \) for all \( d\in D \). Let \( d_1,d_2\in D \), and consider the commutator of \( X_{d_1}, Y_{d_2} \) as a map \( \tau:D\times D \to M^M \):
\begin{equation*}
  \tau(d_1,d_2) = Y_{-d_2}\circ X_{-d_1}\circ Y_{d_2}\circ X_{d_1}
\end{equation*}

Let us recall lemma \ref{lm:3coequ}, which, in summary, states that for any \( g:D\times D \to M \) such that

\begin{equation*}
g(d_1,0) = g(0,d_2) = g(0,0) \quad \forall d_1,d_2\in D
\end{equation*}

there exists a unique map \( t:D\to M \) with the property

\begin{equation*}
  g(d_1,d_2) = t(d_1d_2)
\end{equation*}

The same conclusion applies when instead we have \( g:D\times D\to M^M \), as the only property of \( M \) used in the proof is that it is microlinear. It will be the case that \( \tau \) satisfies those conditions:

\begin{align*}
  \tau(d_1,0) &= Y_0\circ X_{-d_1}\circ Y_0\circ X_{d_1} \\
              &= X_{-d_1}\circ X_{d_1}                   \\
	      &= \id_M = \tau(0,0)                       \\
  \tau(0,d_2) &= Y_{-d_2}\circ X_0\circ Y_{d_2}\circ X_0 \\
              &= Y_{-d_2}\circ Y_{d_2}                   \\
	      &= \id_M = \tau(0,0)                       \\
\end{align*}

Therefore there exists a unique \( t:D\to M \) with \( \tau(d_1,d_2) = t(d_1d_2) \) for all \( d_1,d_2 \). This allows us make the following

\begin{defn}
  Let \( X,Y \) be vector fields on \( M \) (a microlinear object). The \emph{Lie bracket} of \( X \) and \( Y \) is the unique vector field, denoted \( [X,Y] \), such that
  \begin{equation*}
    [X,Y](d_1,d_2) = Y_{-d_2}\circ X_{-d_1}\circ Y_{d_2}\circ X_{d_1}
  \end{equation*}
\end{defn}

What we'll be pleased to find is that this bracket is bilinear, antisymmetric, and it satisfies the Jacobi identity. In short:

\begin{proposition}
  The pair \( (\vfld M, [\farg,\farg]) \) is a Lie algebra.
\end{proposition}

\begin{proof}
  \begin{itemize}
    \item The Lie bracket is antisymmetric. We only need to verify that \( [X,Y](d) = -[Y,X](d) \) for \( d = d_1d_2 \), for some \( d_1,d_2\in D \), by virtue of the uniqueness property of \( [\farg,\farg] \).
      \begin{align*}
	[X,Y]_{d_1d_2} &= [X,Y]_{-(-d_1d_2)}                                                   \\
		       &= \left( [X,Y]_{-d_1d_2} \right)^{-1}                                  \\
		       &= \left( Y_{-d_2}\circ X_{d_1}\circ Y_{d_2}\circ X_{-d_1} \right)^{-1} \\
		       &= X_{d_1}\circ X_{-d_2}\circ X_{-d_1}\circ Y_{d_2}                     \\
		       &= [Y,X]_{d_2(-d_1)}                                                    \\
		       &= \left( -[Y,X] \right)_{d_1d_2}                                       \\
      \end{align*}
    \item Again we utilize that \( \vfld M \) is Euclidean to prove linearity, by only proving homogeneity (prop. \ref{prop:homg}). Let \( \lambda\in R \). Then
      \begin{align*}
	\lambda[X,Y]_{d_1d_2} &= [X,Y]_{\lambda d_1d_2}                                               \\
	                      &= Y_{-d_2}\circ X_{-\lambda d_1}\circ Y_{d_2}\circ X_{\lambda d_1}     \\
			      &= Y_{-d_2}\circ (\lambda X)_{-d_1}\circ Y_{d_2}\circ (\lambda X)_{d_1} \\
			      &= [\lambda X,Y]_{d_1d_2}
      \end{align*}
    \item It remains to prove the Jacobi identity:
      \begin{equation*}
	[[X,Y],Z] + [[Y,Z],X] + [[Z,X],Y] = 0
      \end{equation*}
      for all vector fields \( X,Y,Z \). We will not prove it here, but the idea is the same as in the previous proofs. First note that it suffices to prove it on elements \( d_1d_2d_3 \). You then expand the left hand side to
      \begin{align*}
	&Z_{-d_3}\circ X_{-d_1}\circ Y_{-d_2}\circ X_{d_1}\circ Y_{d_2}\circ Z_{d_3}\circ Y_{-d_2}\circ X_{-d_1}\circ Y_{d_2}\circ X_{d_1} \\
        &\circ X_{-d_1}\circ Y_{-d_2}\circ Z_{-d_3}\circ Y_{d_2}\circ Z_{d_3}\circ X_{d_1}\circ Z_{-d_3}\circ Y_{-d_2}\circ Z_{d_3}\circ Y_{d_2} \\
        &\circ Y_{-d_2}\circ Z_{-d_3}\circ X_{-d_1}\circ Z_{d_3}\circ X_{d_1}\circ Y_{d_2}\circ X_{-d_1}\circ Z_{-d_3}\circ X_{d_1}\circ Z_{d_3} 
      \end{align*}
      and make a series of judicious manipulations to reach \( 0(d_1d_2d_3 = \id_M \). The complete proof can be found in \cite{lav96}.
  \end{itemize}
\end{proof}

\section{Derivations}
 The final construction we'll look at is that of the Lie derivative associated with a vector field. Its definition is much the same as the classical one; we evaluate functions \( M\to R \) along the flow induced by a vector field. Recall from the previous section that a vector field \( X \) is equivalent to giving a function
 \begin{equation*}
   X:M\times D\to M
 \end{equation*}

Hence, given a function \( f:M\to R \) the mapping
\begin{equation*}
  f(X(p,\farg):D\to R
\end{equation*}
is determined uniquely by \( f(X(p,0)) = f(p) \), and some \( b\in R \) with
\begin{equation*}
  f(X(p,d)) = f(p) + b(p)d
\end{equation*}
Here \( p\in M \) is fixed, and \( b \) depends on it (and \( f \), and \( X \)), as we've made explicit. Varying \( p \) gives a function \( M\to R \), leading to this

\begin{defn}
  Let \( f:M\to R \) be a function, and \( X\in \vfld M \). The Lie derivative of \( f \) along \( X \) is the unique function
  \begin{equation*}
    \lie_Xf:M\to R
  \end{equation*}
  such that
  \begin{equation*}
    f(X(p,d)) = f(p) + d\cdot(\lie_Xf)(p)
  \end{equation*}
\end{defn}

The Lie derivative is a derivation, as we will see shortly. We should call to memory that a \emph{derivation} of an algebra \( A \) is a linear map \( D:A\to A \) such that \( D(fg) = D(f)g + fD(g) \) for all \( f,g\in A \).

\begin{proposition}
  The map given by associating each function \( f:M\to R \) to its Lie derivative along \( X \) is a derivation of \( R \)-algebras.
\end{proposition}

\begin{proof}
  Homogeneity, and therefore linearity, is a given, since
  \begin{align*}
    (\lambda\lie_Xf)(p)    &= \lambda f(p) + d\cdot\lambda(\lie_Xf)(p)      \\
    (\lie_X(\lambda f))(p) &= (\lambda f)(p) + d\cdot(\lie_X(\lambda f))(p) \\
                           &= \lambda f(p) + d\cdot(\lie_X(\lambda f))(p)
  \end{align*}
  and by uniqueness \( \cdot\lambda(\lie_Xf)(p) = d\cdot(\lie_X(\lambda f))(p) \). The ``product rule'' will follow from the same proof we gave of the calculus version in section \ref{sec:basictheory}. Let \( f,g:M\to R \) be functions. The Lie derivative along \( X \) is defined by the expression
  \begin{align*}
    (fg)(X(p,d)) &= f(X(p,d))g(X(p,d))                                     \\
                 &= (f(p) + d\cdot(\lie_Xf)(p))(g(p) + d\cdot(\lie_Xg)(p)) \\
		 &= f(p)g(p) + d((\lie_Xf)(p)g(p) + f(p)(\lie_Xg)(p))
  \end{align*}
  and the unique coefficient multiplying \( d \) is clearly the same as for \( \lie_Xf\cdot g + f\cdot\lie_Xg \).
\end{proof}

\chapter{A quick overview of models}

The exposition of the material so far is, as the introduction states, axiomatic. Consider the analogy with a first course in differential calculus; the real numbers most likely simply ``given'' as some variation of ``a complete Archimedean ordered field'' that exists (or nothing is stated, and properties are simply invoked when needed). The task of actually exhibiting such an object is left until a later point in time, presumably when the student has matured enough to grasp the necessary material.

Here we have done the same, albeit in a much higher context. To carry on with the theory we have instated a pair \( (\sdgE,R) \), where \( \sdgE \) is a topos and \( R \) is a \( \Q \)-algebra of \( \sdgE \) which satisfies the Kock-Lawvere axiom. Two concerns arise:
\begin{itemize}
  \item Does there exist such a pair in the standard mathematical universe?
  \item If so, how do we associate results in SDG to their classical counterparts?
\end{itemize}

Naturally, both questions have been answered. For instance, in \cite[§III]{kock06} the author constructs models which answer the first question (the answer is ``yes''). The nature of these models is outside the scope of the present work, but pose an excellent subject for further study (in effect, it would be quite necessary if one were interested in pursuing SDG at a professional level).

The second question is answered by what are referred to as \emph{well-adapted} models of SDG. More work has to be done to exhibit such a model, but again this has been accomplished and one can find it in the literature, for example in the same \cite{kock06}. Once more, the details make heavy use of category theory at an advanced level, but we can describe some of the attractive properties that well-adapted models possess, in regards to comparing SDG with classical geometry.

Let \( \manf \) be the category of Hausdorff paracompact \( \cinf \) manifolds. To give a well-adapted model of SDG is to exhibit a category \( \sdgE \), and a full and faithful functor
\begin{equation*}
  i:\manf\to\sdgE
\end{equation*}
which satisfies certain properties. We won't go into detail about those properties (see \cite{kock06}), but we can look at some of the consequences of such a model. For instance, \( \sdgE \) has an object \( i(\R) \), and this will be our \( R \). It satisfies the Kock-Lawvere axiom for one, and furthermore there are such comparison results such as the following. Given a smooth map \( f:\R\to\R \), through \( i \) we obtain a map \( i(f):R\to R \). It is the case (if \( (i,\sdgE) \) is a well-adapted model), that
\begin{equation*}
  (i(f))' = i(f')
\end{equation*}
In other words, derivatives in the standard sense correspond exactly with derivatives in SDG. It's also true that \( i \) preserves products, so that \( \R^n \) gets taken to \( R^n \), and given a function \( f:\R^n\to R \), then
\begin{equation*}
  i\left( \ddx{f}{x_j} \right) = \ddx{i(g)}{x_j}
\end{equation*}

Another property of well-adapted models is that they preserve tangent bundles. Formally, if \( M \) is a manifold in \( \manf \), then there exists a natural isomorphism
\begin{equation*}
  i(TM)\stackrel{\alpha_M}{\to} (i(M))^D = T(i(M))
\end{equation*}

With these properties one can now be assured that SDG is well suited to reasoning and drawing conclusions in ``standard'' differential geometry. One particularly well detailed example of the procedure for doing so can be found in \cite{bun17}, where the authors give a synthetic proof of the Ambrose-Palais-Singer theorem, along with a synthetic treatment of (parts of) the calculus of variations - in particular the characterization of geodesic curves as critical curves of the action integral.


\backmatter


\huge{
  \bibliographystyle{plain}
  \bibliography{main}
}

\end{document}
